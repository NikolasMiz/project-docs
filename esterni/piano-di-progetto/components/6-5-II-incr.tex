\documentclass[../piano-di-progetto.tex]{subfiles}

\begin{document}

\subsection{II incremento}

Di seguito vengono riportate le ore di lavoro effettive durante il periodo del secondo incremento:
\begin{table}[H]
    \centering
    \begin{tabular}{lccccccc}
      \rowcolor{lightgray}
      \textbf{Nominativo}       & \textbf{Re}      & \textbf{Am} & \textbf{An}      & \textbf{Pt} & \textbf{Pr} & \textbf{Ve} & \textbf{Ore totali} \\
Sofia Bononi              & -          & -               & -          & -          & 5                & -           & 5                \\
Enrico Buratto            & 2          & -               & -          & 1          & -                & 2           & 5                \\
Ian Nicolas Di Menna      & -          & -               & 3          & -          & -                & 2           & 5                \\
Alessandro Franchin       & -          & 3 (+1)          & -          & -          & -                & 2           & 5 (+1)           \\
Enrico Galdeman           & -          & -               & -          & -          & 7 (+2)           & -           & 7 (+2)           \\
Nicholas Miazzo           & -          & -               & -          & 2          & -                & 3           & 5                \\
Marco Nardelotto          & -          & -               & -          & 3          & -                & 1           & 4                \\
\textbf{Ore totali ruolo} & \textbf{2} & \textbf{3 (+1)} & \textbf{3} & \textbf{6} & \textbf{12 (+2)} & \textbf{10} & \textbf{35 (+3)}

    \end{tabular}
    \caption{Resoconto orario effettivo del periodo del secondo incremento}
  \end{table}

  \begin{table}[H]
    \centering
    \begin{tabular}{lcccccc}
      \rowcolor{lightgray}
      \textbf{Ruolo}  & \textbf{Ore previste} & \textbf{Ore effettive} & \textbf{Costo previsto} & \textbf{Costo effettivo} & \textbf{Differenza} \\
Responsabile    & 2           & 2           & € 60,00           & € 0,00            & € 0,00           \\
Amministratore  & 2           & 3           & € 40,00           & € 60,00           & € 20,00          \\
Analista        & 3           & 3           & € 75,00           & € 0,00            & € 0,00           \\
Progettista     & 6           & 6           & € 132,00          & € 0,00            & € 0,00           \\
Programmatore   & 10          & 10          & € 150,00          & € 180,00          & € 30,00          \\
Verificatore    & 10          & 10          & € 150,00          & € 0,00            & € 0,00           \\
\textbf{Totale} & \textbf{33} & \textbf{34} & \textbf{€ 607,00} & \textbf{€ 657,00} & \textbf{€ 50,00}

    \end{tabular}
    \caption{Resoconto economico effettivo del periodo del secondo incremento}
  \end{table}


\subsubsection{Conclusioni}
Questo periodo ha richiesto tre in più rispetto al monte ore preventivato, in particolare:
\begin{itemize}
    \item \textbf{Amministratore}: questo ruolo ha richiesto un'ora in più a causa di una rivisitazione del processo di CI;
    \item \textbf{Programmatore}: lo sviluppo dell'interfaccia grafica del programma di addestramento si è rivelato più complesso del previsto a causa della necessità di implementare alcuni controlli per facilitare l'utilizzo all'utente;
\end{itemize}
In conclusione, in questo periodo, sono stati spesi € 50,00 in più rispetto al preventivo.

\subsubsection{Preventivo a finire}
Il bilancio economico viene chiuso in positivo di € 35,00, considerando i € 15,00 risparmiati nei periodi precedenti. Si stima di poter recuperare l'eccesso di ore durante il quarto periodo in quanto l'attenta attività di pianificazione andrà a facilitare di molto l'integrazione delle componenti. 
\end{document}
