\documentclass[../piano-di-progetto.tex]{subfiles}
\begin{document}

\subsection{Analisi}
Il periodo di analisi inizia il 2020-03-09 con la formazione del gruppo e termina il giorno 2020-04-13 con la consegna dei documenti per la Revisione dei Requisiti. 

\subsubsection{Ruoli}
Durante questa macro, viene richiesta la presenza dei seguenti ruoli:
\begin{itemize}
    \item Responsabile;
    \item Amministratore;
    \item Analista;
    \item Verificatore.
\end{itemize}

\subsection{Attività}

\begin{itemize}
    \item \textbf{I periodo ()}:
        \begin{itemize}
            \item \textbf{Discussione capitolati}: analisi e discussione dei capitolati proposti al fine di trovare gli aspetti positivi e negativi di ciascuno e individuare i capitolati di maggior interesse;
            \item \textbf{Ricerca degli strumenti}: ricerca degli strumenti di supporto da utilizzare durante le attività di progetto;
            \item \textbf{Normazione}: definizione delle Norme di progetto per i processi di supporto e organizzativi;
            \item \textbf{Pianificazione attività e ruoli} pianificazione e assegnazione delle attività preliminari;
            \item \textbf{Studio di fattibilità}: studio di tutti i capitolati;
            \item \textbf{Pianificazione della qualità}: definizione delle metriche per garantire la qualità di processo;
            \item \textbf{Verifica}.
        \end{itemize}
        \item \textbf{II periodo ()}:
            \begin{itemize}
                \item \textbf{Verifica disponibilità proponenti}: le aziende proponenti vengono contattate per la conferma, o variazione, delle disponibilità dei capitolati;
                \item \textbf{Scelta capitolato}: decisione definitiva del capitolato da sviluppare;
                \item \textbf{Studio di fattibilità}: approfondimento dello studio di fattibilità del capitolato scelto;
                \item \textbf{Normazione}: definizione delle Norme di progetto per i processi primari;
                \item \textbf{Pianificazione della qualità}: definizione delle metriche per garantire la qualità di prodotto;
                \item \textbf{Analisi dei rischi}: analisi approfondita dei rischi che il gruppo potrebbe riscontare durante lo svolgimento del progetto e definizione del relativo piano di contingenza;
                \item \textbf{Glossario}: aggiornamento del glossario;
                \item \textbf{Verifica};
            \end{itemize}
        \item \textbf{III periodo ()}:
        \begin{itemize}
            \item \textbf{Normazione}: completamento al dettaglio delle norme di progetto;
            \item \textbf{Analisi dei casi d'uso};
            \item \textbf{Pianificazione delle attività}: pianificazione delle attività future e rispettivo preventivo;
            \item \textbf{Pianificazione della qualità}: specifica dei test;
            \item \textbf{Verifica};
        \end{itemize}
        \item \textbf{IV periodo ()}:
        \begin{itemize}
            \item \textbf{Analisi dei requisiti};
            \item \textbf{Resoconto}: stesura del consultivo di periodo e analisi dei rischi riscontrati;
            \item \textbf{Ricerca degli strumenti}: ricerca e studio autonomo degli strumenti richiesti dal proponente per lo sviluppo del progetto;
            \item \textbf{Glossario};
            \item \textbf{Verifica};

        \end{itemize}

        \item \textbf{V periodo ()}:
        \begin{itemize}
            \item \textbf{Stesura lettera di presentazione}: stesura della lettere di presentazione per la prima revisione di avanzamento;
            \item \textbf{Ricerca degli strumenti}: ricerca e studio autonomo degli strumenti richiesti dal proponente per lo sviluppo del progetto;
            \item \textbf{Verifica};
        \end{itemize}
    \end{itemize}


\end{document}