\begin{document}

\subsection*{IDE}
\addcontentsline{toc}{subsection}{IDE}
Un IDE ovvero Integrated Development Environment, in informatica, è un software che, in fase di programmazione, supporta i programmatori nello sviluppo del codice sorgente di un programma, per esempio segnalando errori di sintassi del codice durante la fase di scrittura o offrendo funzionalità come il debugging.

\subsection*{Incremento}
\addcontentsline{toc}{subsection}{Incremento}
Inserimento di nuove funzionalità o caratteristiche.

\subsection*{Indice di Gulpease}
\addcontentsline{toc}{subsection}{Indice di Gulpease}
L'Indice Gulpease è un indice di leggibilità di un testo tarato sulla lingua italiana. Rispetto ad altri ha il vantaggio di utilizzare la lunghezza delle parole in lettere anziché in sillabe, semplificandone il calcolo automatico.

\subsection*{Infrastructure-as-a-service}
\addcontentsline{toc}{subsection}{Infrastructure-as-a-service}
Infrastructure-as-a-service, abbreviato comunemente in IaaS, sono servizi online che forniscono API di alto livello.

\subsection*{Inspection}
\addcontentsline{toc}{subsection}{Inspection}
Tecnica che si basa sulla lettura mirata dei documenti/codice in cerca di errori specifici.

\subsection*{Issue}
\addcontentsline{toc}{subsection}{Issue}
Una issue è costituita da un problema che impedisce o rallenta l’avanzamento di un progetto e che necessita di un lavoro aggiuntivo da parte del project manager, del team di progetto e di eventuali esperti esterni per trovare una soluzione che consenta di superarlo o risolverlo.

\subsection*{Issue Tracking System}
\addcontentsline{toc}{subsection}{Issue Tracking System}
Strumento che facilita la gestione del processo di sviluppo attraverso la gestione di attività diverse.
Ogni singola attività del progetto è gestita mediante un workflow e mantenuta all’interno di un’unica piattaforma e di un’unica repository.
\end{document}
