\begin{document}
\subsection*{Baseline}
\addcontentsline{toc}{subsection}{Baseline}
In generale è una linea di base per un progetto. Questa può comprendere i requisiti, le motivazioni che stanno dietro alla scelta di tecnologie, framework e librerie per lo sviluppo del progetto.

\subsection*{Blockchain}
\addcontentsline{toc}{subsection}{Blockchain}
La Blockchain è una struttura dati condivisa e immutabile. È definita come un registro digitale le cui voci sono raggruppate in blocchi, concatenati in ordine cronologico, e la cui integrità è garantita dall'uso della crittografia.

\subsection*{Branch}
\addcontentsline{toc}{subsection}{Branch}
Un branch indica lo sviluppo di un nuovo progetto software, o di una nuova componente di esso, che parte da del codice sorgente già esistente. Nello specifico del progetto, un branch rappresenta una diramazione specifica propria di un componente del progetto, sia esso di natura software o documentativa.

\subsection*{Branching}
\addcontentsline{toc}{subsection}{Branching}
Indica l'attività di creazione di diversi branch all'interno di un progetto.

\subsection*{Briefing}
\addcontentsline{toc}{subsection}{Briefing}
Un briefing è una breve riunione di lavoro, all'interno della quale vengono impartiti ordini operativi o vengono scambiate informazioni riguardanti lo stato del compito a cui ognuno sta lavorando.
\end{document}
