\begin{document}
\subsection*{Open Source}
\addcontentsline{toc}{subsection}{Open Source}
In informatica, il termine Open Source viene utilizzato per riferirsi ad un tipo di software o al suo modello di sviluppo o distribuzione. Un software open source è reso tale per mezzo di una licenza attraverso cui i detentori dei diritti favoriscono la modifica, lo studio, l'utilizzo e la redistribuzione del codice sorgente.

\subsection*{Orientato agli oggetti}
\addcontentsline{toc}{subsection}{Orientato agli oggetti}
La programmazione orientata agli oggetti è un paradigma di programmazione, che prevede di raggruppare in un'unica entità (la classe) sia le strutture dati che le procedure che operano su di esse, creando quindi un oggetto.
\end{document}
