\begin{document}
\subsection*{Git}
\addcontentsline{toc}{subsection}{Git}
Git è un software di controllo versione distribuito utilizzabile da interfaccia a riga di comando.

\subsection*{Git Commitizen}
\addcontentsline{toc}{subsection}{Git Commitizen}
Git Commitizen è un progetto Open Source che permette di formattare in modo semplice i messaggi per i commit così che seguano le convenzioni.

\subsection*{GitHub}
\addcontentsline{toc}{subsection}{GitHub}
GitHub è un servizio di hosting per progetti software. È una implementazione dello strumento di controllo versione distribuito Git.

\subsection*{GitHub Actions}
\addcontentsline{toc}{subsection}{GitHub Actions}
Le Git Hub Actions aiutano ad automatizzare il flusso di lavoro dello sviluppo del software presente nella Git Hub repository.

\subsection*{GitHub Issues}
\addcontentsline{toc}{subsection}{GitHub Issues}
Le Git Hub Issues vengono utilizzate per tener traccia dei compiti, migliorie e bug del proprio progetto.

\subsection*{Git Submodules}
\addcontentsline{toc}{subsection}{Git Submodules}
Un submodule, come dice la documentazione ufficiale di Git, è un repository git inserito nel working tree di un altro repository git, come dipendenza.

\subsection*{Google Chrome}
\addcontentsline{toc}{subsection}{Google Chrome}
Google Chrome, detto anche semplicemente Chrome, è un browser web sviluppato da Google, basato sul motore di rendering Blink (a partire dalla versione 28, precedentemente sfruttava WebKit).
Basato sul browser Chromium, Chrome, nel corso degli anni, è cresciuto a tal punto da diventare il browser più usato al mondo nell'aprile 2016 con una percentuale del 41,81\% secondo il sito Netmarketshare.

\subsection*{Grafana}
\addcontentsline{toc}{subsection}{Grafana}
Grafana è un software multipiattaforma open source che provvede diagrammi, grafici e allarmi per poter analizzare i dati. È espandibile tramite un sistema di plug-in.

\subsection*{Grafico}
\addcontentsline{toc}{subsection}{Grafico}
Un grafico è la rappresentazione grafica di un fenomeno, di un ente o di un sistema matematico.

\subsection*{GUI}
\addcontentsline{toc}{subsection}{GUI}
L'interfaccia grafica (nota anche come GUI (dall'inglese Graphical User Interface), in informatica è un tipo di interfaccia utente che consente l'interazione uomo-macchina in modo visuale utilizzando rappresentazioni grafiche (es. widget) piuttosto che utilizzando i comandi tipici di un'interfaccia a riga di comando.
\end{document}
