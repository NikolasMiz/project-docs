\begin{document}
\subsection*{Machine Learning}
\addcontentsline{toc}{subsection}{Machine Learning}
L’apprendimento automatico (noto anche come machine learning) è una branca dell'intelligenza artificiale che raccoglie e utilizza un insieme di metodi statistici per migliorare progressivamente la performance di un algoritmo nell'identificare pattern nei dati.

\subsection*{Macro-processo}
\addcontentsline{toc}{subsection}{Macro-processo}
Indica un processo "macro" (esteso), ovvero un grande processo composto da altri processi minori.

\subsection*{Major release}
\addcontentsline{toc}{subsection}{Major release}
Una Major release specifica la versione di un prodotto che ne determina l’incremento di uno dei numeri di versione più significativi (dipendente dalle specifiche e dal contesto ove è definito), a seguito di modifiche importanti. Ne determina la distribuzione.

\subsection*{Manutenzione predittiva}
\addcontentsline{toc}{subsection}{Manutenzione predittiva}
La manutenzione predittiva è un tipo di manutenzione che viene effettuata a seguito dell'individuazione di uno o più parametri che vengono misurati ed elaborati utilizzando modelli matematici, allo scopo di individuare il tempo residuo prima del guasto.

\subsection*{Markup, linguaggio di}
\addcontentsline{toc}{subsection}{Markup, linguaggio di}
Il linguaggio di markup è un insieme di regole che descrivono i meccanismi di rappresentazione (strutturali, semantici, presentazionali) o layout di un testo; facendo uso di convenzioni rese standard, tali regole sono utilizzabili su più supporti.

\subsection*{Metriche}
\addcontentsline{toc}{subsection}{Metriche}
Una metrica è uno standard per la misura di alcune proprietà del software o delle sue specifiche.

\subsection*{Micro-servizi}
\addcontentsline{toc}{subsection}{Micro-servizi}
I microservizi sono piccoli servizi autonomi che interagiscono tra di loro e che hanno come finalità quella di svolgere un compito.

\subsection*{Milestone}
\addcontentsline{toc}{subsection}{Milestone}
Traguardo intermedio ed importante nello svolgimento di un progetto; sono spesso fissati in fase di pianificazione.

\subsection*{Minor release}
\addcontentsline{toc}{subsection}{Minor release}
Una Minor release specifica la versione di un prodotto che ne determina l’incremento di uno dei numeri di versione meno significativi (dipendente dalle specifiche e dal contesto ove è definito), a seguito di modifiche minori al prodotto.

\subsection*{Modello}
\addcontentsline{toc}{subsection}{Modello}
L'oggetto o il termine atto a fornire un conveniente schema di punti di riferimento ai fini della riproduzione o dell'imitazione.

\subsection*{Modello di sviluppo}
\addcontentsline{toc}{subsection}{Modello di sviluppo}
In informatica un modello di sviluppo software è il principio teorico che indica il metodo da seguire nel progettare e nello scrivere un programma.

\subsection*{Motore di rendering}
\addcontentsline{toc}{subsection}{Motore di rendering}
Un motore di rendering, in informatica ed in particolare nella computer grafica, è un componente hardware o software che interpreta delle informazioni in ingresso codificate secondo uno specifico formato e le elabora creandone una rappresentazione grafica. 
\end{document}
