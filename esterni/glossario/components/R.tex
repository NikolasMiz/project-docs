\begin{document}

\subsection*{Recall}
\addcontentsline{toc}{subsection}{Recall}
La recall è una frazione del totale delle istanze rilevanti che vengono individuate in un insieme di dati.

\subsection*{Regressione lineare}
\addcontentsline{toc}{subsection}{Regressione lineare}
La regressione lineare è quella tecnica statistica utilizzata per studiare le relazioni che intercorrono tra due o più caratteri (variabili) statistici.

\subsection*{Repository}
\addcontentsline{toc}{subsection}{Repository}
Un repository (letteralmente deposito o ripostiglio), in informatica, è un ambiente di un sistema informativo  in cui vengono gestiti i metadati, attraverso tabelle relazionali; l'insieme di tabelle, regole e motori di calcolo tramite cui si gestiscono i metadati prende il nome di metabase.

\subsection*{Requisiti}
\addcontentsline{toc}{subsection}{Requisiti}
I requisiti sono delle qualità necessarie e richieste per uno scopo determinato.

\subsection*{Responsabile}
\addcontentsline{toc}{subsection}{Responsabile}
Il responsabile è la figura cardine per quanto concerne le responsabilità di pianificazione, controllo, gestione e coordinamento di risorse all’interno del progetto. Vedere sezione 4.1.4.1 del documento \textsc{Norme di Progetto}.

\subsection*{Responsive}
\addcontentsline{toc}{subsection}{Responsive}
Il design responsive, letteralmente design responsivo, indica una tecnica di web design per la realizzazione di siti in grado di adattarsi graficamente in modo automatico al dispositivo coi quali vengono visualizzati

\subsection*{Reti neurali}
\addcontentsline{toc}{subsection}{Reti neurali}
Le reti neurali sono modelli matematici composti da neuroni artificiali che si ispirano al funzionamento biologico del cervello umano.
Risultano indispensabili per risolvere problemi ingegneristici di Intelligenza Artificiale e richiedono avanzati chip hardware a supporto.

\subsection*{Road map}
\addcontentsline{toc}{subsection}{Road map}
Roadmap (dall'inglese "mappa stradale") è comunemente intesa come una sequenza temporale di azioni previste attraverso la quale ci si aspetta di raggiungere un obiettivo. Generalmente definisce il piano di sviluppo di un nuovo prodotto o di una nuova tecnologia.
\end{document}
