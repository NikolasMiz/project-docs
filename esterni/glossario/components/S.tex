\begin{document}
\subsection*{Sage Maker}
\addcontentsline{toc}{subsection}{Sage Maker}
Sage Maker è una piattaforma di machine learning cloud lanciata a novembre 2017. Sage Maker consente agli sviluppatori di creare, formare e distribuire modelli di machine learning nel cloud.

\subsection*{Serverless}
\addcontentsline{toc}{subsection}{Serverless}
Un network serverless è un network la cui gestione non viene incentrata su server, bensì viene dislocata fra i vari utenti che utilizzano tale rete. Il lavoro di gestione del network necessario viene quindi eseguito dagli stessi utilizzatori.

\subsection*{Skype}
\addcontentsline{toc}{subsection}{Skype}
Skype è un software proprietario freeware di messaggistica istantanea e VoIP.

\subsection*{Slack (strumento di lavoro)}
\addcontentsline{toc}{subsection}{Slack (strumento di lavoro)}
Slack è un software che rientra nella categoria degli strumenti di collaborazione aziendale utilizzato per inviare messaggi in modo istantaneo ai membri di un team lavorativo.

\subsection*{Slack (tempo di attesa)}
\addcontentsline{toc}{subsection}{Slack (tempo di attesa)}
Letteralmente "tempo allentato", è il tempo di possibile ritardo di un’attività che non comporta
un ritardo per il progetto di cui fa parte.

\subsection*{Smart contracts}
\addcontentsline{toc}{subsection}{Smart contracts}
Gli smart contracts sono protocolli informatici che facilitano, verificano o fanno rispettare la negoziazione o l'esecuzione di un contratto, permettendo talvolta la parziale o totale esclusione di una clausola contrattuale.

\subsection*{Sottoprocessi}
\addcontentsline{toc}{subsection}{Sottoprocessi}
Sono processi che fanno parte di un processo più grande.

\subsection*{Spellchecker}
\addcontentsline{toc}{subsection}{Spellchecker}
Strumento utilizzato per controllare l'ortografia.

\subsection*{Suite}
\addcontentsline{toc}{subsection}{Suite}
indica un insieme di applicazioni che permettono all'utente di un computer di creare dei contenuti quali documenti di testo, grafici o presentazioni, tipicamente ad uso personale o nel lavoro d'ufficio.
Si tratta di strumenti comunemente utilizzati nell'ambito dell'informatica di base.

\subsection*{Support Vector Machines}
\addcontentsline{toc}{subsection}{Support Vector Machines}
Le Support Vector Machines (SVM) sono dei modelli di apprendimento supervisionato associati ad algoritmi di apprendimento per la regressione e la classificazione.
\end{document}
