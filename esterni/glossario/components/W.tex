\begin{document}
\subsection*{Walkthrough}
\addcontentsline{toc}{subsection}{Walkthrough}
Tecnica che consiste in una lettura del documento/codice cercando errori ed anomalie a largo spettro senza un’idea precisa di quali tipi errori sarà possibile trovare.

\subsection*{Way of working}
\addcontentsline{toc}{subsection}{Way of working}
Insieme  di  processi  di  progetto  istanziati  per  una  particolare  fase  del  ciclo  di  vita  del
progetto. Definisce  per  costruzione  la  conformità  del  prodotto  agli  obiettivi  di
qualità e funzionalità identificati.

\subsection*{Workflow}
\addcontentsline{toc}{subsection}{Workflow}
Con il termine Workflow (tradotto letteralmente "flusso di lavoro") si intende la creazione di modelli e la gestione informatica dell'insieme dei compiti e i diversi attori coinvolti nella realizzazione di un processo lavorativo

\subsection*{Workspace}
\addcontentsline{toc}{subsection}{Workspace}
Si riferisce all’ambiente di lavoro nel quale è possibile creare e gestire più progetti.
\end{document}
