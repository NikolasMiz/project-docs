\begin{document}

\subsection*{Account}
\addcontentsline{toc}{subsection}{Account}
In informatica, indica il complesso dei dati identificativi di un utente, che gli consentono l'accesso a un servizio telematico.

\subsection*{Addestramento}
\addcontentsline{toc}{subsection}{Addestramento}
In informatica, si intende l'atto di fornire dati a un'intelligenza artificiale con lo scopo di costruire un modello probabilistico generale dello spazio delle occorrenze.

\subsection*{Alert}
\addcontentsline{toc}{subsection}{Alert}
Genericamente un segnale o un messaggio (finestra) di avviso, allarme, avvertimento o conferma.

\subsection*{Amazon Web Services}
\addcontentsline{toc}{subsection}{Amazon Web Services}
Amazon Web Services è un'azienda statunitense appartenente al gruppo amazon che fornisce servizi di cloud computing.

\subsection*{Amministratore}
\addcontentsline{toc}{subsection}{Amministratore}
L’amministratore è la figura che gestisce e controlla l’ambiente di lavoro. Vedere sezione 4.1.4.2 del documento \textsc{Norme di Progetto}.

\subsection*{Analista}
\addcontentsline{toc}{subsection}{Analista}
L’analista è colui che si occupa di tutte le attività di analisi. Vedere sezione 4.1.4.3 del documento \textsc{Norme di Progetto}.

\subsection*{Angular}
\addcontentsline{toc}{subsection}{Angular}
Angular è un framework open source per lo sviluppo di applicazioni web con licenza MIT, evoluzione di AngularJS.
Sviluppato principalmente da Google.

\subsection*{Apache Kafka}
\addcontentsline{toc}{subsection}{Apache Kafka}
Apache Kafka è una piattaforma open source di stream processing, scritta in Java e in Scala, sviluppata dall'Apache foundation.

\subsection*{API}
\addcontentsline{toc}{subsection}{API}
API è l'acronimo di Application Programming Interface. Le API sono insiemi di definizioni e protocolli con i quali vengono integrati software applicativi.

\subsection*{API Restful}
\addcontentsline{toc}{subsection}{API Restful}
Un set di API che rispettano il modello architetturale REST.

\subsection*{Applicazioni decentralizzate}
\addcontentsline{toc}{subsection}{Applicazioni decentralizzate}
Un'applicazione decentralizzata è un'applicazione con codice di back-end in esecuzione su una rete peer-to-peer decentralizzata.

\subsection*{Applicazioni web}
\addcontentsline{toc}{subsection}{Applicazioni web}
Un'applicazione web consiste in una applicazione accessibile e/o fruibile via web per mezzo di un network, cioè in una architettura tipica di tipo client-server.

\subsection*{Architettura}
\addcontentsline{toc}{subsection}{Architettura}
Con architettura si intende l'insieme di criteri di progetto in base ai quali è progettato e realizzato il prodotto appartenete al progetto.

\subsection*{AWS Rekognition video}
\addcontentsline{toc}{subsection}{AWS Rekognition video}
Un servizio di analisi video in grado di riconoscere i movimenti delle persone in un fotogramma, e di riconoscere soggetti, volti, oggetti e altri contenuti nei video.

\subsection*{AWS Transcode}
\addcontentsline{toc}{subsection}{AWS Transcode}
AWS Transcode è un servizio offerto da Amazon Web Services per la transcodifica di contenuti multimediali in cloud.

\end{document}
