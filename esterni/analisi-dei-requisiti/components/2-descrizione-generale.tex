\documentclass[../analisi-dei-requisiti.tex]{subfiles}

\begin{document}

\subsection{Obiettivo del prodotto}
\label{subs:obiettivo_del_prodotto}
Il progetto \emph{Predire in Grafana} si pone come obiettivo finale la realizzazione di un plug-in per il sistema di monitoraggio Grafana. Il plug-in deve essere accompagnato con un programma esterno per l'addestramento per i \glossario{modelli} di machine learning. Grazie a questo, esso sarà in grado di adattarsi ai diversi flussi di dati e riuscirà a fornire delle previsioni sulle possibili \glossario{criticità} riscontrate sulla linea di produzione del software. Gli utenti del software potranno analizzare la situazione tramite grafici e dashboard, facilitando la visualizzazione di problematiche anche gravi all'interno della linea di produzione.

\subsection{Funzioni del prodotto}
\label{subs:funzioni_del_prodotto}
	Le funzioni offerte dal prodotto sono due.
	La prima riguarda l'addestramento dei modelli di machine learning usati: SVM e RL, che avviene tramite l'applicazione di addestramento. L'addestramento deve essere effettato dall'utente che deve:
	\begin{itemize}
		\item{Creare un file in formato \glossario{CSV} per fornire un set di dati utili e significativi;}
		\item{Aprire il file tramite un software per fogli elettronici, come ad esempio LibreOffice oppure Microsoft Excel, e assegnargli la classificazione corretta;}
		\item{Caricare il file nell'applicazione di addestramento;}
		\item{Se si è in possesso di un file già processato dall'applicazione allora esso può essere caricato nuovamente con un formato \glossario{JSON} per ottenere un aggiornamento dei valori in base ai nuovi dati;}
		\item{Scegliere il tipo di modello che si desidera addestrare tra SVM o RL;}
		\item{Se si sceglie RL allora è possibile indicare un'ulteriore trasformazione ai dati,scegliendo la funzione tra quelle disponibili;}
		\item{Avviare la procedura di addestramento;}
		\item{Infine si deve salvare il file in formato JSON con il risultato dell'addestramento e il modello utilizzato.}
	\end{itemize}
	Dopo aver finito l'addestramento del modello, l'utente deve eseguire i seguenti passi per configurare in modo adeguato il plug-in su Grafana:
	\begin{itemize}
		\item{Aprire la pagina delle impostazioni di Granafa e abilitare il plug-in;}
		\item{Aprire la pagina di configurazione dei plug-in;}
		\item{Selezionare il file JSON con la definizione del modello usato e i risultati dell'addestramento, ovvero il predittore da usare;}
		\item{Scegliere tra i nodi disponibili quelli che si desidera aggiungere alla rete di flusso di monitoraggio sui quali verranno effettuate le previsioni;}
		\item{Scegliere quale delle due modalità di visualizzazione dei dati generati dalle previsioni usare:}
		\begin{itemize}
			\item{\textbf{Indicatore}: permette di visualizzare in maniera compatta se in un nodo è stata prevista una criticità;}
			\item{\textbf{Grafico}: permette di visualizzare in un grafico l'evoluzione delle previsioni, in cui sarà presente l'indicatore di soglia massima.}
		\end{itemize}
		In entrambi i casi verrà fornita o visualizzata la bontà della previsione.
		\item{Infine si può avviare la predizione.}
	\end{itemize}
	Alla fine del procedimento l'utente viene reindirizzato alla dashboard creata dal plug-in contenente la previsione effettuata nei rispettivi pannelli.

\subsection{Caratteristiche degli utenti}
\label{subs:caratteristiche_degli_utenti}
  Gli utenti che possono utilizzare il plug-in per monitorare un flusso di dati sono soltanto coloro che sono autenticati in Grafana. Un utente non autenticato non potrà usufruire del plug-in in quanto questo non può esistere come entità autonoma, ma solo come estensione per Grafana.

\subsection{Architetture e Tecnologie del progetto}
\label{subs:architetture_e_tecnologie_del_progetto}

	\subsubsection{Architettura del prodotto}
	\label{sssec:architettura_del_prodotto}
		Il prodotto prevede la realizzazione di un plug-in in linguaggio \glossario{JavaScript} in quanto Grafana, piattaforma che lo utilizzerà per individuare i punti deboli del proprio sistema grazie a previsioni che indicano quali siano le aree con maggiori criticità da risolvere, è espandibile attraverso un sistema di plug-in.

	\subsubsection{Programma di addestramento}
	\label{sssec:programma_di_addestramento}
		Il prodotto deve essere accompagnato da un programma di addestramento che permetta di eseguire le previsioni corrette adeguandosi ai diversi flussi di dati monitorati dall'utente. Per entrambi i modelli di machine learning si dovrà partire da una base di dati noti che consenta di capire la correlazione tra i dati e i valori delle previsioni attese. Per fare questo è necessario disporre di un' applicazione esterna che permetta all'utente di addestrare i due modelli per produrre previsione di qualità sui propri dati secondo il funzionamento spiegato nella sezione 2.2 di codesto documento.

	\subsubsection{Plug-in di previsione}
	\label{sssec:plug-in_di_previsione}
		Il plug-in che verrà realizzato dal team caricherà il file JSON, prodotto dai dati di addestramento con i parametri per le previsioni con SVM per la classificazione o la RL, leggerà la definizione del predittore e successivamente assocerà tali predittori al flusso dei dati presenti in Grafana. In seguito i risultati verranno resi disponibili al sistema di creazione di grafici e dashboard per la loro visualizzazione. Il sistema, grazie a previsioni accompagnate da motivazioni e dal livello di qualità del valore ottenuto, permetterà al sistemista di analizzare la situazione e verificare se è necessario un intervento o se si tratti di una situazione sopravvalutata.

subsection{Vincoli generali}
\label{subs:vincoli_generali}
	L'implementazione deve rispettare i seguente vincoli:
	\begin{itemize}
		\item{Utilizzo del linguaggio JavaScript per lo sviluppo del plug-in.}
		\begin{itemize}
			\item{Il proponente consiglia l'uso di una libreria creata da Davide Ghiotto, presente al seguente indirizzo:
			\newline \href{https://www.npmjs.com/package/ml-modules}{Libreria per il machine learning.}}
			\item{Il proponente consiglia anche l'uso di due librerie per l'implementazione dei modelli d'addestramento SVM e RL rispettivamente nei seguenti indirizzi:
			\newline \href{https://github.com/karpathy/svmjs}{Libreria in JavaScript per le SVM;}
			\newline \href{https://github.com/Tom-Alexander/regression-js}{Libreria in JavaScript per la RL.}}
			\item{Per le \glossario{reti neurali} in JavaScript, il proponente consiglia l'uso della libreria presente al seguente link:
			\newline \href{https://cd.standford.edu/people/karpathy/convnetis}{Libreria in JavaScript per le reti neurali.}}
		\end{itemize}
		\item{L'addestramento dei modelli SVM e RL deve essere eseguito in un applicazione esterna a cui vengono forniti i dati di test}
		\begin{itemize}
			\item{L'esito dell' addestramento deve essere salvato in un file JSON con i parametri per le previsioni con SVM o RL per le classificazioni.}
		\end{itemize}
	\item{Il plug-in deve essere in grado di svolgere i seguenti compiti:}
		\begin{itemize}
			\item{Produrre un file json dai dati di addestramento con i parametri per le previsioni con SVM per le classificazioni o la RL;}
			\item{Leggere la definizione del predittore dal file in formato JSON;}
			\item{Associare i predittori letti dal file JSON al flusso di dati presente in Grafana;}
			\item{Applicare la previsione e fornire al sistema di Grafana i nuovi dati ottenuti dalla previsione;}
			\item{Rendere disponibili i dati al sistema di creazione di grafici e dashboard per la loro visualizzazione.}
		\end{itemize}
	\end{itemize}
	I requisiti opzionali sono:
	\begin{itemize}
		\item{Possibilità di definire "\glossario{alert}" in base ai livelli di soglia raggiunti dai nodi collegati alle previsioni;}
		\item{Fornire i dati di bontà dei modelli di previsione: "Precision" e "Recall" per le SVM e "R\^{}2" per la RL;}
		\item{Possibilità di applicare delle trasformazioni alle misure lette dal campo per ottenere delle regressioni esponenziali o logaritmiche e non solo lineari;}
		\item{Possibilità di addestrare la SVM o RL direttamente su Grafana;}
		\item{Implementare dei meccanismi di apprendimento di flusso, in modo da poter disporre di sistemi di previsione in costante adattamento ai dati rilevati sul campo;}
		\item{Utilizzare anche altri metodi di previsione, tra cui la versione delle SVM adattate alla Regressione o piccole reti neurali per la classificazione.}
	\end{itemize}

\end{document}
