\documentclass[../analisi-dei-requisiti.tex]{subfiles}

\begin{document}

\subsection{Struttura}
I requisiti sono identificati secondo uno schema univoco; questo è così composto: \\
\begin{center}
  \centering
  \textbf{R[T]-[P]-[I]}
\end{center} dove:
\begin{itemize}
  \item \textbf{R}: Indica che il codice è un requisito;
  \item \textbf{T}: Indica la tipologia del requisito. Il requisito può essere:
  \begin{itemize}
    \item \textbf{F}: Funzionale;
    \item \textbf{Q}: Qualitativo;
    \item \textbf{V}: Vincolo.
  \end{itemize}
  \item \textbf{P}: Indica la priorità del requisito. Questa può essere:
  \begin{itemize}
    \item \textbf{O}: Indica un requisito obbligatorio poiché necessario;
    \item \textbf{D}: Indica un requisito desiderabile, ma non necessario;
    \item \textbf{F}: Indica un requisito opzionale o contrattabile in corso d'opera del progetto.
  \end{itemize}
  \item \textbf{I}: Indica l'identificativo del requisito nella forma \\
  \begin{center}
    \centering
    \textbf{codicePadre.codiceFiglio}
  \end{center} dove entrambi i codici sono numeri progressivi.
\end{itemize}

\subsection{Requisiti di funzionalità}
\label{sub:requisiti_di_funzionalita}

\rowcolors{2}{white!80!lightgray!90}{white}
\renewcommand{\arraystretch}{2} % allarga le righe con dello spazio sotto e sopra
\begin{longtable}[H]{>{\centering\bfseries}m{2cm} >{\centering}m{9cm} >{\centering}m{2.5cm} >{\centering\arraybackslash}m{2.5cm}}
  \caption{Requisiti funzionali}%
  \label{tab:requisiti_funzionali}                                                    \\
  \rowcolor{lightgray}
  {\textbf{Requisito}} & {\textbf{Descrizione}} & {\textbf{Priorità}} & {\textbf{Fonte}}  \\
  \endfirsthead%
  \rowcolor{lightgray}
  {\textbf{Requisito}} & {\textbf{Descrizione}} & {\textbf{Priorità}} & {\textbf{Fonte}}  \\
  \endhead%
  \rowcolor{white}
  \multicolumn{4}{c}{\textit{Continua alla pagina successiva}}
  \endfoot%
  \endlastfoot%
  \textbf{RFO1} & Il sistema deve essere addestrato dall'amministratore. & Obbligatorio & UC1 \\
  \textbf{RFO1.1} & Il sistema deve mostrare un pulsante che permetta all'amministratore di caricare il file CSV contenente i dati di addestramento.  & Obbligatorio & UC1.1 \\
  \textbf{RFD1.1.1} & Il sistema potrebbe notificare un messaggio di errore nel caso in cui venga fornito un file di addestramento non valido. & Desiderabile & UC8 \\
  \textbf{RFO1.2} & Il sistema deve mostrare un pulsante che permetta all'amministratore di caricare il predittore allenato in precedenza, se ne è in possesso. & Obbligatorio & UC1.2 \\
  \textbf{RFD1.2.1} & Il sistema potrebbe notificare all'amministratore un messaggio d'errore nel caso in cui si tenti di caricare un predittore in formato non valido. & Desiderabile & UC16 \\
  \textbf{RFO1.3} & Il sistema deve permettere all'amministratore di scegliere un modello di machine learning da utilizzare per l'allenamento. & Obbligatorio & UC1.3 \\
  \textbf{RFO1.3.1} & Il sistema deve permettere all'amministratore di selezionare il modello SVM per l'allenamento. & Obbligatorio & UC1.3.1 \\
  \textbf{RFO1.3.2} & Il sistema deve permettere all'amministratore di selezionare il modello RL per l'allenamento. & Obbligatorio & UC1.3.2 \\
  \textbf{RFF1.3.2.1} & Il sistema può permettere all'amministratore di selezionare la trasformazione esponenziale per l'allenamento. & Facoltativo & UC1.3.2.1 \\
  \textbf{RFF1.3.2.2} & Il sistema può permettere all'amministratore di selezionare la trasformazione logaritmica per l'allenamento. & Facoltativo & UC1.3.2.2 \\
  \textbf{RFF1.3.3} & Il sistema può permettere all'amministratore di selezionare il modello Rete Neurale per l'allenamento. & Facoltativo & UC1.3.3 \\
  \textbf{RFO1.4} & Il sistema deve permettere all'amministratore di avviare l'addestramento. & Obbligatorio & UC1.4 \\
  \textbf{RFO1.5} & Il sistema deve mettere a disposizione all’amministratore un pulsante per il caricamento del file JSON contenente il predittore allenato in precedenza. & Obbligatorio & UC1.5 \\
  \textbf{RFF2} & Il sistema può mettere a disposizione all'amministratore un metodo per l’addestramento direttamente in Grafana. & Facoltativo & UC2 \\
  \textbf{RFF2.1} & Il sistema può permettere all’amministratore di avviare l’addestramento in Grafana. & Facoltativo & UC2 \\
  \textbf{RFF2.2} & Il sistema può permettere all'amministratore di selezionare un flusso di dati a cui applicare la predizione. & Facoltativo & UC2.1 \\
  \textbf{RFF2.4} & Il sistema può permettere all’amministratore di selezionare il modello di previsione desiderato. & Facoltativo & UC2.2 \\
  \textbf{RFO3} & Il sistema deve permettere all’amministratore di configurare il plug-in. & Obbligatorio & UC3 \\
  \textbf{RF03.1} & Il sistema deve mettere a disposizione all’amministratore un metodo per caricare il modello addestrato. & Obbligatorio & UC3.1 \\
  \textbf{RFD3.1.1} & Il sistema potrebbe notificare un messaggio d'errore nel caso in cui venga caricato un file contenente il modello addestrato non valido. & Desiderabile & UC9 \\
  \textbf{RFD3.1.2} & Il sistema potrebbe notificare un messaggio d'errore nel caso in cui il non venga caricato nessun file. & Desiderabile & UC10 \\
  \textbf{RFO3.2} & Il sistema deve consentire all’Amministratore di selezionare i nodi su cui desidera effettuare la predizione. & Obbligatorio & UC3.2 \\
  \textbf{RFD3.2.1} & Il sistema potrebbe notificare un messaggio d'errore nel caso in cui venga selezionato un nodo non valido. & Desiderabile & UC11 \\
  \textbf{RFD3.2.2} & Il sistema potrebbe notificare un messaggio d'errore nel caso in cui non venga selezionato alcun nodo. & Desiderabile & UC12 \\
  \textbf{RFO3.3} & Il sistema deve permettere all’amministratore la selezione del tipo di visualizzazione della predizione. & Obbligatorio & UC3.3 \\
  \textbf{RFO3.3.1} & Il sistema deve mettere a disposizione la possibilità di visualizzare la previsione utilizzando un indicatore. & Obbligatorio & UC3.3.1 \\
  \textbf{RFO3.3.2} & Il sistema deve mettere a disposizione la possibilità di visualizzare la previsione utilizzando un grafico. & Obbligatorio & UC3.3.2 \\
  \textbf{RFD3.3.3} & Il sistema potrebbe notificare un messaggio d'errore nel caso in cui non venga selezionato alcun tipo di visualizzazione. & Desiderabile & UC13 \\
  \textbf{RFO4} & Il sistema deve mettere a disposizione dell’utente un metodo per l’avvio e l’interruzione della predizione. & Obbligatorio & UC4 \\
  \textbf{RFO4.1} & Il sistema deve mettere a disposizione dell’amministratore un pulsante per l’avvio della predizione. & Obbligatorio & UC4.1 \\
  \textbf{RFO4.2} & Il sistema deve mettere a disposizione dell’amministratore un pulsante per interrompere della predizione. & Obbligatorio & UC4.2 \\
  \textbf{RFF5} & Il sistema può mettere a disposizione dell’amministratore un metodo per l’impostazione degli alert. & Facoltativo & UC5 \\
  \textbf{RFF5.1} & Il sistema può mostrare un pulsante che permetta all'amministratore di creare un alert. & Facoltativo & UC5.1 \\
  \textbf{RFF5.2} & Il sistema può mettere a disposizione un sistema per inserire una soglia di alert. & Facoltativo & UC5.2 \\
  \textbf{RFF5.2.1} & Il sistema può mettere a disposizione dell’amministratore un pulsante per confermare la soglia dell’alert desiderata. & Facoltativo & UC5.2 \\
  \textbf{RFD5.2.1.1} & Il sistema, utilizzando un alert, potrebbe notificare un messaggio di errore nel caso in cui venga superata la soglia impostata. & Desiderabile & UC14 \\
  \textbf{RFD6} & Il sistema potrebbe permettere all'utente di eliminare il pannello di monitoraggio. & Desiderabile & UC6 \\
  \textbf{RFD6.1} & Il sistema potrebbe permettere all'utente di selezionare il pannello di monitoraggio da rimuovere. & Desiderabile & UC6 \\
  \textbf{RFD6.2} & Il sistema potrebbe permettere grazie alle impostazione la rimozione del pannello di monitoraggio. & Desiderabile & UC6 \\
  \textbf{RFD6.3} & Il sistema potrebbe visualizzare un messaggio di errore nel caso in cui si cerchi di eliminare il pannello senza interrompere la previsione. & Desiderabile & UC15 \\
  \textbf{RFO7} & Il sistema deve mettere a disposizione all'utente un metodo per la visualizzazione delle previsioni. & Obbligatorio & UC7 \\


\end{longtable}


\newpage

\subsection{Requisiti di qualità}
\label{sub:requisiti_di_qualita}

\rowcolors{2}{white!80!lightgray!90}{white}
\renewcommand{\arraystretch}{2} % allarga le righe con dello spazio sotto e sopra
\begin{longtable}[H]{>{\centering\bfseries}m{2cm} >{\centering}m{9cm} >{\centering}m{2.5cm} >{\centering\arraybackslash}m{2.5cm}}
  \caption{Requisiti di qualità}%
  \label{tab:requisiti_di_qualità}                                                    \\
  \rowcolor{lightgray}
  {\textbf{Requisito}} & {\textbf{Descrizione}} & {\textbf{Priorità}} & {\textbf{Fonte}}  \\
  \endfirsthead%
  \rowcolor{lightgray}
  {\textbf{Requisito}} & {\textbf{Descrizione}} & {\textbf{Priorità}} & {\textbf{Fonte}}  \\
  \endhead%
  \rowcolor{white}
  \multicolumn{4}{c}{\textit{Continua alla pagina successiva}}
  \endfoot%
  \endlastfoot%
  \textbf{RQO1} & Bisogna fornire un manuale utente per l'uso del plug-in sviluppato in formato pdf. & Obbligatorio & Capitolato \\
  \textbf{RQO1.1} & Il manuale utente deve essere in lingua italiana. & Obbligatorio & Decisione interna \\
  \textbf{RQD1.2} & Il manuale utente deve essere in lingua inglese. & Desiderabile & Decisione interna \\
  \textbf{RQO2} & Bisogna fornire un manuale sviluppatore per l'estensibilità del prodotto in formato pdf. & Obbligatorio & Capitolato \\
  \textbf{RQO2.1} & Il manuale sviluppatore deve essere in lingua italiana. & Obbligatorio & Decisione interna \\
  \textbf{RQD2.2} & Il manuale sviluppatore deve essere in lingua inglese. & Desiderabile & Decisione interna \\
  \textbf{RQO3} & Il codice sorgente del plug-in sviluppato deve essere disponibile in una repository pubblica su "github.com" o in altre repository pubbliche. & Obbligatorio & Capitolato \\
  \textbf{RQO4} & Lo sviluppo del plug-in deve essere conforme alle norme stabilite nel documento \textsc{NORME DI PROGETTO}. & Obbligatorio & Decisione interna \\
  \textbf{RQO5} & Lo sviluppo del plug-in deve essere conforme ai vincoli stabiliti nel documento \textsc{PIANO DI QUALIFICA}. & Obbligatorio & Decisione interna \\
\end{longtable}


\newpage

\subsection{Requisiti di vincolo}
\label{sub:requisiti_di_vincolo}

\rowcolors{2}{white!80!lightgray!90}{white}
\renewcommand{\arraystretch}{2} % allarga le righe con dello spazio sotto e sopra
\begin{longtable}[H]{>{\centering\bfseries}m{2cm} >{\centering}m{9cm} >{\centering}m{2.5cm} >{\centering\arraybackslash}m{2.5cm}}
  \caption{Requisiti di vincolo}%
  \label{tab:requisiti_di_vincolo}                                                    \\
  \rowcolor{lightgray}
  {\textbf{Requisito}} & {\textbf{Descrizione}} & {\textbf{Priorità}} & {\textbf{Fonte}}  \\
  \endfirsthead%
  \rowcolor{lightgray}
  {\textbf{Requisito}} & {\textbf{Descrizione}} & {\textbf{Priorità}} & {\textbf{Fonte}}  \\
  \endhead%
  \rowcolor{white}
  \multicolumn{4}{c}{\textit{Continua alla pagina successiva}}
  \endfoot%
  \endlastfoot%
  \textbf{RVO1} & Il codice del plug-in deve essere scritto in linguaggio ES6. & Obbligatorio & Grafana Developer Guide \\
  \textbf{RVO2} & Va usato un sistema di build che supporti systemjs. & Obbligatorio & Grafana Developer Guide \\
  \textbf{RVO3} & Il file da usare nel plug-in che contiene l'entry point del codice dovrà essere chiamato module.js. & Obbligatorio & Grafana Developer Guide \\
  \textbf{RVO4} & Il codice prodotto dal plug-in e la pagina d'addestramento deve essere open source & Obbligatorio. & Capitolato \\
  \textbf{RVO5} & La parte del software per l'allenamento del file da inserire su Grafana deve essere una pagina web esterna. & Obbligatorio & Capitolato \\
  \textbf{RVO5.1} & La pagina web d'addestramento deve essere scritta usando HTML5 e JavaScript. & Obbligatorio & Proponente \\
  \textbf{RVO5.2} &	L'allenamento del predittore per il modello RL deve utilizzare la libreria fornita dal proponente. & Obbligatorio & Proponente \\
  \textbf{RVO5.3} &	L'allenamento del predittore per il modello SVM deve utilizzare la libreria fornita dal proponente. & Obbligatorio & Proponente \\

\end{longtable}


\newpage

\newpage


\subsection{Tracciamento}
\label{sub:tracciamento}

\subsubsection{Fonte - Requisiti}
\label{sssec:fonte_requisiti}

\rowcolors{2}{white!80!lightgray!90}{white}
\renewcommand{\arraystretch}{2} % allarga le righe con dello spazio sotto e sopra
\begin{longtable}[H]{>{\centering\bfseries}m{8cm}  >{\centering\arraybackslash}m{8cm}}
  \caption{Fonte - Requisiti}%
  \label{tab:fonte_requisiti}                                                    \\
  \rowcolor{lightgray}
  {\textbf{Fonte}} & {\textbf{Requisito}}  \\
  \endfirsthead%
  \rowcolor{lightgray}
  {\textbf{Fonte}} & {\textbf{Requisito}}  \\
  \endhead%
  \rowcolor{white}
  \multicolumn{2}{c}{\textit{Continua alla pagina successiva}}
  \endfoot%
  \endlastfoot%
  \textbf{Capitolato} & {\begin{tabular}[c]{@{}c@{}}RQO1 \\ RQO2 \\ RQO3 \\ RVO4 \\ RVO5 \end{tabular}} \\
  \textbf{Decisione Interna} & {\begin{tabular}[c]{@{}c@{}}RQO1.1 \\ RQD1.2 \\ RQO2.1 \\ RQD 2.2 \\ \end{tabular}}  \\
  \textbf{Grafana Developer Guide} & {\begin{tabular}[c]{@{}c@{}} RVO1 \\ RVO2 \\ RVO3 \end{tabular}} \\
  \textbf{Proponente} & {\begin{tabular}[c]{@{}c@{}} RVO5.1 \\ RVO5.2 \\ RVO5.3 \end{tabular}} \\
  \textbf{UC1} & {\begin{tabular}[c]{@{}c@{}}RFO1 \\ RFO1.1 \\ RFO1.2 \\ RFO1.3 \\ RFO1.3.1 \\ RFO1.3.2 \\ RFF1.3.2.1 \\ RFF1.3.2.2 \\ RFF1.3.3 \\ RFO1.4 \\ RFO1.5\end{tabular}}\\
  \textbf{UC2} & {\begin{tabular}[c]{@{}c@{}}RFF2 \\ RFF2.1 \\ RFO2.2 \\ RFD2.3 \\ RFD2.4\end{tabular}}\\
  \textbf{UC3} & {\begin{tabular}[c]{@{}c@{}}RFO3 \\ RFO3.1 \\ RFD3.1.1 \\ RFD3.1.2 \\ RFO3.2 \\ RFD3.2.1 \\ RFD3.2.2 \\ RFO3.3 \\ RFO3.3.1 \\ RFO3.3.2 \\ RFD3.3.3\end{tabular}}\\
  \textbf{UC4} & {\begin{tabular}[c]{@{}c@{}}RFO4 \\ RFO4.1 \\ RFO4.2\end{tabular}}\\
  \textbf{UC5} & {\begin{tabular}[c]{@{}c@{}}RFF5 \\ RFF5.1 \\ RFF5.2 \\ RFF5.2.1\end{tabular}}\\
  \textbf{UC6} & {\begin{tabular}[c]{@{}c@{}}RFD6 \\ RFD6.1 \\ RFD6.2 \\ RFD6.3\end{tabular}}\\
  \textbf{UC7} & RFO7 \\
  \textbf{UC8} & RFD1.1.1 \\
  \textbf{UC9} & RFD3.1.1 \\
  \textbf{UC10} & RFD3.1.2 \\
  \textbf{UC11} & RFD3.2.1 \\
  \textbf{UC12} & RFD3.2.2 \\
  \textbf{UC13} & RFD3.3.3 \\
  \textbf{UC14} & RFD5.2.1.1 \\
  \textbf{UC15} & RFD6.3 \\
  \textbf{UC16} & RFD1.2.1 \\

\end{longtable}


\subsubsection{Requisiti - Fonte}
\label{sssec:requisiti_fonte}

\rowcolors{2}{white!80!lightgray!90}{white}
\renewcommand{\arraystretch}{2} % allarga le righe con dello spazio sotto e sopra
\begin{longtable}[H]{>{\centering\bfseries}m{8cm} >{\centering\arraybackslash}m{8cm}}
  \caption{Requisiti - Fonte}%
  \label{tab:requisiti_fonte}                                                    \\
  \rowcolor{lightgray}
  {\textbf{Requisiti}} & {\textbf{Fonte}}  \\
  \endfirsthead%
  \rowcolor{lightgray}
  {\textbf{Requisiti}} & {\textbf{Fonte}}  \\
  \endhead%
  \rowcolor{white}
  \multicolumn{2}{c}{\textit{Continua alla pagina successiva}}
  \endfoot%
  \endlastfoot%
  \textbf{RFO1} & UC1 \\
  \textbf{RFO1.1} & UC1.1 \\
  \textbf{RFD1.1.1} & UC17 \\
  \textbf{RFO1.2} & UC1.2 \\
  \textbf{RFD1.2.1} & UC25 \\
  \textbf{RFO1.3} & UC1.3 \\
  \textbf{RFO1.4} & UC1.4 \\
  \textbf{RFO1.5} & UC1.5 \\
  \textbf{RFF2} & UC2 \\
  \textbf{RFF2.1} & UC2 \\
  \textbf{RFF2.2} & UC2.1 \\
  \textbf{RFF2.3} & UC2.2 \\
  \textbf{RFO3} & UC3 \\
  \textbf{RFO3.1} & UC3.1 \\
  \textbf{RFD3.1.1} & UC18 \\
  \textbf{RFD3.1.2} & UC19 \\
  \textbf{RFO3.2} & UC3.2 \\
  \textbf{RFD3.2.1} & UC20 \\
  \textbf{RFD3.2.2} & UC21 \\
  \textbf{RFO3.3} & UC3.3 \\
  \textbf{RFD3.3.1} & UC22 \\
  \textbf{RFO4} & UC4 \\
  \textbf{RFO5} & UC5 \\
  \textbf{RFO6} & UC6 \\
  \textbf{RFO7} & UC7 \\
  \textbf{RFF8} & UC8 \\
  \textbf{RFF9} & UC9 \\
  \textbf{RFF10} & UC10 \\
  \textbf{RFF11} & UC11 \\
  \textbf{RFO12} & UC12 \\
  \textbf{RFO13} & UC13 \\
  \textbf{RFF14} & UC14 \\
  \textbf{RFF14.1} & UC14.1 \\
  \textbf{RFF14.2} & UC14.2 \\
  \textbf{RFD14.2.1} & UC23 \\
  \textbf{RFD15} & UC15 \\
  \textbf{RFD15.1} & UC24 \\
  \textbf{RFO16} & UC16 \\

\end{longtable}


\subsection{Riepilogo}
\label{sub:riepilogo}

\rowcolors{2}{white!80!lightgray!90}{white}
\renewcommand{\arraystretch}{2} % allarga le righe con dello spazio sotto e sopra
\begin{longtable}[H]{>{\centering\bfseries}m{3cm} >{\centering}m{3cm} >{\centering}m{3cm} >{\centering}m{3cm} >{\centering\arraybackslash}m{3cm}}
  \caption{Riepilogo}%
  \label{tab:riepilogo}                                                    \\
  \rowcolor{lightgray}
  {\textbf{Tipologia}} & {\textbf{Obbligatorio}} & {\textbf{Facoltativo}} & {\textbf{Desiderabile}} & {\textbf{Totale}} \\
  \endfirsthead%
  \rowcolor{lightgray}
  {\textbf{Tipologia}} & {\textbf{Obbligatorio}} & {\textbf{Facoltativo}} & {\textbf{Desiderabile}} & {\textbf{Totale}}   \\
  \endhead%
  \rowcolor{white}
  \multicolumn{5}{c}{\textit{Continua alla pagina successiva}}
  \endfoot%
  \endlastfoot%
  \textbf{Funzionale} & 18 & 11 & 12 & 41 \\
  \textbf{Qualità} & 7 & 0 & 2 & 9 \\
  \textbf{Vincolo} & 8 & 0 & 0 & 8 \\

\end{longtable}


\end{document}
