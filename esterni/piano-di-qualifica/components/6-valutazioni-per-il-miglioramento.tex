\documentclass[../piano-di-qualifica.tex]{subfiles}

\begin{document}
Questa sezione elenca i problemi riscontrati nel corso del progetto definiti dai membri di Coff
L'obiettivo principale è esporre i problemi che sono sorti in questa prima parte di lavoro e di conseguenza valutare quali soluzioni sono state messe in atto per evitare il ripetersi di queste difficoltà.
I principali problemi riscontrati riguardano:
\begin{itemize}
    \item \textbf{L'organizzazione}: problemi inerenti all'organizzazione del gruppo e alla comunicazione interna ad esso;
    \item \textbf{I ruoli}: problemi relativi alla divisione e al corretto svolgimento dei ruoli;
    \item \textbf{Gli strumenti di lavoro}: problemi relativi alla scelta e all'utilizzo degli strumenti.
\end{itemize}
La rilevazione di questi errori e le relative soluzioni sono state rese possibili grazie alla buona comunicazione all'interno del gruppo, migliorata durante il tempo di svolgimento della prima parte del progetto.
Non essendoci una figura esterna al gruppo in grado di fornire una valutazione oggettiva e che possa gestirlo si è resa fondamentale l'autovalutazione di ciascun componente del gruppo e la totale liberta di esprimere le proprie difficoltà e di collaborare a trovare una soluzione comune per risolvere tali problematiche.
Di seguito verranno trattati i problemi attualmente riscontrati, questa sezione verrà costantemente aggiornata nel caso si dovessero riscontrare ulteriori problematiche e attività di miglioramento necessarie per la maturazione del team.

\subsection{Valutazioni sull’organizzazione}
\label{sub:valutazioni_organizzazione}
Vengono definiti i problemi sull'organizzazione del team, punto fondamentale per l'avvio del progetto e che influenza l'intero suo svolgimento.

\subsubsection{Incontri di gruppo}
\label{sub:incontri_di_gruppo}
Data la situazione creatasi le prime comunicazioni all'interno del gruppo sono risultate complicate e confusionarie data l'impossibilità di un incontro dal vivo con tutti i componenti del gruppo.
Se da una parte gli impegni di molti componenti si erano ridotti da un'altra il non poter avere un incontro dal vivo tra i componenti ha causato problematiche nell'organizzazione del gruppo, dato che era necessario trovare un nuovo metodo per comunicare e accordarsi riuscendo comunque ad accumunare gli impegni di tutti per poter riunire tutto il gruppo.
Queste difficoltà sono emerse solo nel periodo iniziale dello svolgimento del lavoro e sono state risolte in breve tempo mediante l'utilizzo di diversi software di videochiama di gruppo per regolarizzare gli incontri interni.
Gli strumenti utilizzati per risolvere queste problematiche sono stati scelti di comune accordo e la documentazione relativa ad essi è consultabile nelle \textsc{Norme di Progetto}.

\subsubsection{Comunicazioni del gruppo}
\label{sub:comunicazione_del_gruppo}
Dopo aver risolto le problematiche sugli incontro nel gruppo, il problema successivo verificatosi è stato quello di trovare un mezzo di comunicazione per poter organizzare le varie chiamate ed eventuali altri incontri specifici su certi prodotti da produrre, tra solo una piccola parte dei componenti del gruppo.
Si è quindi deciso che oltre all'utilizzo di \glossario{Telegram} come strumento di comunicazione istantanea all'interno del team, di usare \glossario{Slack} per la suddivisione dei messaggi in diversi ambiti in modo da non ottenere una sovrapposizione di messaggi di diverso genere come può succede su Telegram che viene comunque utilizzato per comunicazioni "minori".

\subsubsection{Comunicazioni con il proponente}
\label{sub:comunicazione_con_il_proponente}
L'ultima problematica venuto a crearsi a cause della situazione creatasi è stata la difficolta iniziale di come comunicare con il proponente per ottenere informazioni riguardo il Capitolato e per avere un dialogo costante con esso.
Il problema è stato risolto in breve termine, decidendo di utilizzare le email per la comunicazione con il proponente e di utilizzare \glossario{Skype} per le comunicazioni in videochiamata più formali.

\subsection{Valutazioni sui ruoli}
\label{sub:valutazioni_ruoli}
Vengono definiti i problemi relativi alla suddivisione dei ruoli nel gruppo e su eventuali difficoltà nell'adempimento dei compiti relativi al proprio ruolo.
%%% da finire %%%
\subsubsection{Verificatore}
\label{sub:verificatore}
Data l'inesperienza da parte dei \textit{Verificatori} non tutti gli errori presenti nei documenti sono stati immediatamente individuati il che ha comportato la presenza di contraddizioni di contenuto e struttura rispetto tra i diversi documenti.
Come soluzione il gruppo si è impegnato a dedicare una parte del tempo per analizzare meglio insieme i contenuti dei singoli documenti, in modo da renderli adeguati alle norme utilizzate e soprattutto evitando contraddizioni tra i diversi documenti.

\subsection{Valutazioni sugli strumenti}
\label{sub:valutazioni_strumenti}
Vengono definiti i problemi relativi agli strumenti utilizzati dal gruppo per la gestione delle attività interne ed esterne con il proponente.

\subsubsection{Github/Git}
\label{sub:github}
A causa della configurazione della \glossario{repository} alcuni membri del gruppo non conoscevano le specifiche di utilizzo del comando \textit{git cz} per il commit dei file con Fit. Il problema si è facilmente risolto attraverso la lettura della documentazione del comando e grazie anche a un veloce incontro di spiegazione con i membri del gruppo che avevano già familiarità con il comando.

\subsubsection{Latex}
\label{sub:latex}
Il problema principale è stato che non tutti i componenti del gruppo avevano delle buone conoscenze su latex e hanno avuto difficoltà iniziali nel capire come deve essere utilizzato, ci sono poi stati problemi più generici riguardanti la creazione di tabelle e l'inserimento di immagini nei documenti.
Per risolvere tale problema i membri del gruppo che avevano una maggiore conoscenza di latex hanno affiancato gli altri membri aiutandoli a capire e insegnandogli le basi di latex. Una volta appreso meglio il suo funzionamento per ulteriori dubbi o errori è stato fatto affidamento alle ricerche in internet per ottenere ulteriori informazioni e soluzioni in caso di problemi.

\subsubsection{Software di comunicazione con il proponente}
\label{sub:latex}
Come specificato nelle \textsc{Norme di Progetto} il software di comunicazione più usato con il proponente è Skype con il quale però sono stati riscontrati problemi nell'utilizzo sia da parte del team che da parte del proponente, oltra ad altri problemi vari come la pessima qualità streaming che per adesso il servizio offre.
A questo problema non vi è una vera e propria soluzione soddisfacente, il gruppo e il proponente si impegnano a capire meglio come gestire le chiamate col software, mentre per quanto riguarda la parte relativa alla qualità dello streaming non vi è soluzione immediata, questo problema però non influenza più di tanto la comunicazione e la comprensione dei colloqui.

\end{document}