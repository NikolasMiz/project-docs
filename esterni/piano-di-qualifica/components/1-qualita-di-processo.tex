\documentclass[../piano-di-qualifica.tex]{subfiles}

\begin{document}

\subsection{Scopo}%
\label{sub:scopo}
Lo scopo del documento è quello di ricerca la qualità di un prodotto che è soggetta a cambiamenti in base ai processi utilizzati nel corso della realizzazione dello stesso.
Si rende necessario quindi adottare un buon \glossario{ciclo di vita} che determina i processi da utilizzare che dovranno poi essere verificati e valutati attentamente per la corretta riuscita del prodotto.
A tal proposito il gruppo ha scelto di adottare lo standard \glossario{ISO/IEC 15504}, anche noto come \glossario{SPICE}, per garantire la qualità dei processi adottati per la composizione del prodotto e di seguire lo schema del \glossario{Ciclo di Deming} per la gestione iterativa dei controlli e miglioramenti dei processi del prodotto.

\subsection{Nomenclatura metriche e obiettivi di qualità}%
\label{sub:nomenclatura_metriche_e_obiettivi_di_qualita}
Vengono di seguito spiegati gli obiettivi, le metriche e come essi verranno identificati, in modo sintetico:
\begin{itemize}
    \item \textbf{Obiettivi}: 
    \begin{center}
        \centering
        \textbf{QoPR[ID][Nome]}
    \end{center}
        \begin{itemize}
            \item \textbf{QoPR}: sta per "Quality of Process" ossia qualità del processo;
            \item \textbf{ID}: identificatore numerico a 3 cifre;
            \item \textbf{Nome}: riassunto della descrizione del processo.
        \end{itemize}
        Per ogni obiettivo, dopo una breve descrizione, verrà segnalato il codice della metrica che valuterà il superamento dell'obiettivo e verranno inseriti i valori (singolo valore o range di valori) accettabili e desiderabili per definire le soglie imposte per l'obiettivo.
    \item \textbf{Metriche}: 
    \begin{center}
        \centering
        \textbf{MoPR[ID][Nome]}
    \end{center}
        \begin{itemize}
            \item \textbf{MoPR}: sta per "Metric of Process" ossia metrica del processo;
            \item \textbf{ID}: identificatore numerico a 3 cifre;
            \item \textbf{Nome}: riassunto della descrizione della metrica.
        \end{itemize}
\end{itemize}  
Ulteriori informazioni sulle metriche e sul loro funzionamento sono reperibili nel documento Norme di progetto.

\subsection{Processi}%
\label{sub:processi}
Come per metriche e obiettivi, i processi faranno affidamento alla seguente identificazione:
\begin{itemize}
    \item \textbf{Processi}: 
    \begin{center}
        \centering
        \textbf{PRC[ID][Nome]}
    \end{center}
    \begin{itemize}
        \item \textbf{PRC}: sta a indicare "Processo";
        \item \textbf{ID}: identificatore numerico a 3 cifre;
        \item \textbf{Nome}: riassunto della funzione del processo.
    \end{itemize}
\end{itemize}
Verranno poi fornite per ogni processo, come ulteriore arricchimento per le specifiche di ogni progetto:
\begin{itemize}
    \item Le funzioni principali da esso svolte;
    \item Gli obiettivi che vengono ricercati nel processo;
    \item Le metriche adottate, nel caso ce ne fossero, per raggiungere tali obiettivi.
\end{itemize}

\subsubsection{PRC001 Organizzazione e pianificazione del progetto e della sua struttura}%
\label{sub:organizzazione_e_pianificazione_del_progetto_e_della_sua_struttura}
Questo \glossario{macro-processo} ha l'obiettivo di organizzare il lavoro da svolgere per la corretta soddisfazione dei \glossario{requisiti} identificati nel progetto.
In questo processo viene messo in atto il \glossario{Way of Working}, redatto per la consultazione dell'intero \glossario{team di sviluppo} presente nelle \textsc{Norme di progetto}, il quale ha un'importanza significativa perché la corretta attuazione delle regole contenute al suo interno andranno a condizionare la qualità dell'intero progetto.
È quindi opportuno assicurarsi che la pianificazione temporale e i costi sostenuti siano supportati da metriche in grado di verificarne lo stato di avanzamento del progetto confrontandolo con la pianificazione precedentemente prodotta.

\paragraph{Funzioni}
\label{sub:funzioni_1}

\begin{itemize}
    \item \textbf{Sviluppare \glossario{sotto-processi}}: ogni obiettivo prefissato deve essere associato ad azioni precise ognuna delle quali è raggruppata in un sotto-processo;
    \item \textbf{Dividere i compiti}: è fondamentale la corretta suddivisione dei compiti all'interno dei componenti del gruppo CoffeeCode al fine di non creare incomprensioni;
    \item \textbf{Calendarizzare i documenti}: definire delle \glossario{baseline} durante lo svolgimento del progetto;
    \item \textbf{Apprendimento autonomo}: data l'inesperienza del team nell'utilizzo di diverse tecnologie concordate nel \glossario{Capitolato}, sarà richiesto un periodo di formazione personale più lungo. Questo periodo va conteggiato all'interno della calendarizzazione;
    \item \textbf{Standard appropriati}: vengono scelti gli standard più consoni alle finalità del progetto;
    \item \textbf{Stima budget}: bisogna prendere atto delle proprie risorse in termini di costo/persona in modo da rispettare il \glossario{preventivo} stilato nel \textsc{Piano di Progetto}.
\end{itemize}

\paragraph{Metriche}
\label{sub:metriche_1}
\begin{itemize}
    \item \textbf{MoPR001 Varianza dei tempi};
    \item \textbf{MoPR002 Varianza dei costi};
    \item \textbf{MoPR003 Aderenza agli standard};
    \item \textbf{MoPR004 Aderenza ai ruoli};
    \item \textbf{MoPR005 Controllo prodotti}.
\end{itemize}

\paragraph{Obiettivi}
\label{sub:obiettivi_1}
\begin{itemize}
    \item \textbf{QoPR001 Rispetto delle scadenze della pianificazione}: l'obiettivo è rispettare al meglio le scadenza prefissate in modo da non avere rallentamenti durante la realizzazione del progetto;
        \begin{itemize}
            \item \textbf{Metrica di riferimento}: MoPR001;
            \item \textbf{Valore accettabile}: al massimo 2; %definire meglio nelle norme
            \item \textbf{Valore desiderabile}: 0.
        \end{itemize}
    \item \textbf{QoPR002 Rispetto del budget istanziato}: devono potersi mantenere le risorse definite all'inizio del progetto;
        \begin{itemize}
            \item \textbf{Metrica di riferimento}: MoPR002;
            \item \textbf{Valore accettabile}: $\pm$ 5\% rispetto al totale preventivato; %definire meglio nelle norme
            \item \textbf{Valore desiderabile}: corrisponde al preventivo definito nel PianoDiProgetto.
        \end{itemize}
    \item \textbf{QoPR003 Rispetto del ciclo di vita scelto}: ogni fase del Ciclo di Deming deve essere rispettata dai processi;
        \begin{itemize}
            \item \textbf{Metrica di riferimento}: MoPR003;
            \item \textbf{Valore accettabile}: Livello di maturità: 3 - Valutazione attributi: L; %definire meglio nelle norme
            \item \textbf{Valore desiderabile}: Livello di maturità: 3 - Valutazione attributi: L.
        \end{itemize}
    \item \textbf{QoPR004 Rispetto dei ruoli e identificazione nei prodotti}: deve essere possibile riconoscere i ruoli dei componenti nella partecipazione alla realizzazione di un prodotto e le varie modifiche da loro fatte;
        \begin{itemize}
            \item \textbf{Metrica di riferimento}: MoPR004;
            \item \textbf{Valore accettabile}: 0; %definire meglio nelle norme
            \item \textbf{Valore desiderabile}: 0.
        \end{itemize}
    \item \textbf{QoPR005 Rispetto del versionamento dei prodotti}: ogni prodotto realizzato durante i processi deve possedere un numero di versionamento in modo da tracciare le modifiche, e in caso di errori, trovare un punto di ripristino per il prodotto.
        \begin{itemize}
            \item \textbf{Metrica di riferimento}: MoPR005;
            \item \textbf{Valore accettabile}: 15+; %definire meglio nelle norme
            \item \textbf{Valore desiderabile}: 20+.
        \end{itemize}
\end{itemize}

\subsubsection{PRC002 Analisi}
\label{sub:analisi}
In questo processo viene fatta un'analisi più generica, comprendente altri contenuti che non si riferiscono solo all'analisi dei requisiti (consultabile nel documento \textsc{Analisi dei Requisiti}) o all'analisi dei rischi (consultabile nel documento \textsc{Piano di Progetto}), ma che comprende altre attività riportate dalle seguenti funzioni:

\paragraph{Funzioni}
\label{sub:funzioni_2}
\begin{itemize}
    \item \textbf{Individuazione dei requisiti}: il testo del capitolato è presentato in modo più discorsivo che tecnico, vi è quindi la necessita di effettuare un'analisi per individuare requisiti: espliciti ,impliciti e classificarli in seguito;
    \item \textbf{Individuazione delle risorse}: analizzare le risorse e realizzare poi un preventivo è fondamentale, per il \glossario{fornitore} per verificare la fattibilità del progetto e per il clienti per valutare il costo del progetto;
    \item \textbf{Individuazione dei tempi}: capire la durata del progetto e stabilire la scadenza delle varie fasi è importante per gestire le varie risorse senza incombere in perdite di tempo o in ritardi;
    \item \textbf{Individuazione dei rischi}: capire quali rischi possono presentarsi durante la realizzazione del progetto è fondamentale per prevedere e in caso attuare delle reazioni ad essi.
\end{itemize}

\paragraph{Metriche}
\label{sub:metriche_2}
\begin{itemize}
    \item \textbf{MoPR006 Verifica requisiti obbligatori};
    \item \textbf{MoPR007 Verifica requisiti opzionali};
    \item \textbf{MoPR008 Verifica requisiti desiderabili};
    \item \textbf{MoPR009 Verifica rischi non pervenuti}.
\end{itemize}

\paragraph{Obiettivi}
\label{sub:obiettivi_2}
\begin{itemize}
    \item \textbf{QoPR006 Soddisfazione dei requisiti obbligatori}: tutti i requisiti obbligatori, concordati con il cliente, devono essere soddisfatti a fine progetto;
        \begin{itemize}
            \item \textbf{Metrica di riferimento}: MoPR006;
            \item \textbf{Valore accettabile}: 100\%; %definire meglio nelle norme
            \item \textbf{Valore desiderabile}: 100\%.
        \end{itemize}
    \item \textbf{QoPR007 Soddisfazione dei requisiti opzionali e desiderabili}: i requisiti opzionali e desiderabili devono essere definiti in modo corretto, verificati insieme al cliente e possono essere soddisfatti solo se sono state avanzate risorse, in termini di $\frac{tempo}{persona}$, prima della fine del progetto;
        \begin{itemize}
            \item \textbf{Metriche di riferimento}: MoPR007,MoPR008;
            \item \textbf{Valore accettabile}: 30\% (valore totale delle due metriche); %definire meglio nelle norme
            \item \textbf{Valore desiderabile}: 50\% (valore totale delle due metriche).
        \end{itemize}
    \item \textbf{QoPR008 Verifica dei rischi previsti}: per mantenere costante lo sviluppo del progetto non dovrebbero presentarsi rischi non preventivati che potrebbero comportare a ritardi sui tempi e a spreco di risorse.
        \begin{itemize}
            \item \textbf{Metrica di riferimento}: MoPR009;
            \item \textbf{Valore accettabile}: 0; %definire meglio nelle norme
            \item \textbf{Valore desiderabile}: 0.
        \end{itemize}
\end{itemize}

\subsubsection{PRC003 Produzione dei documenti}
\label{sub:produzione_dei_documenti}
Questo processo rimane operativo per tutta la durata del progetto in quanto ha il compito di redigere i vari documenti, necessari nelle diverse fasi di sviluppo, che comprendano le scelte effettuate, gli strumenti utilizzati e le modifiche decise ed inserite durante la durata del progetto.

\paragraph{Funzioni}
\label{sub:funzioni_3}
Il processo segue di pari passo il ciclo di vita di ogni documento, come spiegato nelle NormeDiProgetto e prevede le fasi di:
\begin{itemize}
    \item Redazione;
    \item Modifica;
    \item Verifica;
    \item Approvazione.        
\end{itemize}

\paragraph{Metriche}
\label{sub:metriche_3}
\begin{itemize}
    \item \textbf{MoPR010 Analisi Way of Working};
    \item \textbf{MoPR011 Analisi documenti}.
\end{itemize}

\paragraph{Obiettivi}
\label{sub:obiettivi_3}
\begin{itemize}
    \item \textbf{QoPR09 Rispetto delle fasi del ciclo di vita}: deve essere rispettato ogni fase del ciclo di vita del documento e le scadenze previste;
        \begin{itemize}
            \item \textbf{Metrica di riferimento}: MoPR010;
            \item \textbf{Valore accettabile}: /; %definire meglio nelle norme
            \item \textbf{Valore desiderabile}: /.
        \end{itemize}
    \item \textbf{QoPR010 Rispetto nella redazione dei documenti}: devono essere rispettate le linee guida inserite nelle NormeDiProgetto.
        \begin{itemize}
            \item \textbf{Metrica di riferimento}: MoPR011;
            \item \textbf{Valore accettabile}: 3; %definire meglio nelle norme
            \item \textbf{Valore desiderabile}: 4+.
        \end{itemize}
\end{itemize}

\subsubsection{PRC004 Verifica}
\label{sub:produzione_dei_documenti}
Durante tutto lo svolgimento del progetto, il processo di verifica sarà attivo con l'obiettivo principale quello di valutare la correttezza dei prodotti generati per: 
\begin{itemize}
    \item Individuare eventuali errori;
    \item Decidere se rispettano gli standard di qualità prefissati.
\end{itemize}

\paragraph{Funzioni}
\label{sub:funzioni_4}
\begin{itemize}
    \item \textbf{Verificare i requisiti dei prodotti}: viene verificato che i prodotti soddisfino i requisiti richiesti concordati;
    \item \textbf{Verificare l'aderenza alle Norme di Progetto}: viene verificato il rispetto del Way Of Working, contenuto nelle \textsc{Norme di Progetto}.
\end{itemize}

\paragraph{Metriche}
\label{sub:metriche_4}
\begin{itemize}
    \item \textbf{MoPR012 Frequenza di controlli}.
\end{itemize}

\paragraph{Obiettivi}
\label{sub:obiettivi_4}
\begin{itemize}
    \item \textbf{QoPR011 Attuare una verifica costante}: si effettuano verifiche costanti in ogni fase del progetto, testando e controllando il prodotto ogni volta che viene modificato o che si è in procinto di terminare una \glossario{milestone};
    \begin{itemize}
        \item \textbf{Metrica di riferimento}: MoPR012;
        \item \textbf{Valore accettabile}: /; %definire meglio nelle norme
        \item \textbf{Valore desiderabile}: /.
    \end{itemize}
    \item \textbf{QoPR012 Comunicare costantemente durante la verifica}: mantenere una comunicazione costante con tutti i ruoli all'interno del gruppo;
    \begin{itemize}
        \item \textbf{Metrica di riferimento}: non è presente una metrica in quanto non è un obiettivo facilmente misurabile;
        \item \textbf{Valore accettabile}: /; %definire meglio nelle norme
        \item \textbf{Valore desiderabile}: /.
    \end{itemize}
    \item \textbf{QoPR013 Rispettare le fasi di verifica}: si esegue la fase di verifica in modo iterativo, rispettando sempre ciò che è stato deciso all'interno delle Norme di Progetto, in modo da rendere coerenti i risultati ottenuti.
        \begin{itemize}
            \item \textbf{Metrica di riferimento}: MoPR010 rientra in questa metrica perché l'attività di verifica fa parte del ciclo di vita del prodotto;
            \item \textbf{Valore accettabile}: /; %definire meglio nelle norme
            \item \textbf{Valore desiderabile}: /.
        \end{itemize}
\end{itemize}

\subsection{Tabelle di qualità di processo}
\label{sub:tabelle_di_qualita_di_processo}
Gli obiettivi di qualità, discussi nelle precedenti sezioni, che devono essere parte integrante di ogni processo, verranno indicati in tabelle in questa sezione.
Per ogni obiettivo viene indicato:

\begin{itemize}
   \item \textbf{Obiettivo}: indica il codice identificativo dell'obiettivo come descritto nella sezione §2.2;
   \item \textbf{Metrica}: indica, se presente, la metrica adottata per la valutazione dell'obiettivo di qualità come descritto nella sezione §2.2;
   \item \textbf{Valore accettabile}: rappresenta il valore minimo di qualità dell'obiettivo che CoffeCode intende ottenere. Non è presente in caso di mancanza della metrica associata all'obiettivo;
   \item \textbf{Valore desiderato}: rappresenta il valore di qualità dell'obiettivo che CoffeCode intende ottenere per una maggiore qualità rispetto a quella minima. Non è presente in caso di mancanza della metrica associata all'obiettivo;
   \item \textbf{Descrizione}: descrizione generale dell'obiettivo.
\end{itemize}
%Tabelle

\begin{center}
    \centering
    \textbf{PRC001 Organizzazione e pianificazione del progetto e della sua struttura}
\end{center}
\begin{longtable}[c]{cccc}
    \hline
    Obiettivo &
      Metrica &
      Valore accettabile &
      Valore desiderabili \\ \hline
    \endhead
    %
    \begin{tabular}[c]{@{}c@{}}QoPR001 Rispetto delle scadenze\\  della pianificazione\end{tabular} &
      \begin{tabular}[c]{@{}c@{}}MoPR001 Varianza \\ dei tempi\end{tabular} &
      Massimo 2 &
      0 \\
    \multicolumn{1}{l}{Descrizione:} &
      \multicolumn{3}{l}{\begin{tabular}[c]{@{}l@{}}Per le scadenze che sono riportate nel Piano di Progetto, il team \\ ha considerato di introdurre dei tempi di \glossario{slack},per prevenire\\ eventuali ritardi. Il completamento di un compito, che corrisponde\\ al completamento di una \glossario{issue} e alla sua chiusura, può accumulare\\ un ritardo massimo di 2 giorni.\end{tabular}} \\ \hline
    \begin{tabular}[c]{@{}c@{}}QoPR002 Rispetto  del  budget \\ istanziato\end{tabular} &
      \begin{tabular}[c]{@{}c@{}}MoPR002 Varianza \\ dei costi\end{tabular} &
      $\pm$ 5\% &
      0\% \\
    \multicolumn{1}{l}{Descrizione:} &
      \multicolumn{3}{l}{\begin{tabular}[c]{@{}l@{}}Per ogni ruolo all’interno del gruppo vi è una tariffa oraria fissa.\\ Può però accadere che durante il corso del progetto sia richiesta\\ una mole di lavoro differente da quella preventivata inizialmente. \\ Si è quindi deciso di tollerare un discostamento massimo del 5\%\\ in positivo o negativo.\end{tabular}} \\ \hline
    \begin{tabular}[c]{@{}c@{}}QoPR003 Rispetto del \\ ciclo di vita scelto\end{tabular} &
      \begin{tabular}[c]{@{}c@{}}MoPR003 Aderenza\\ agli standard\end{tabular} &
      \begin{tabular}[c]{@{}c@{}}Livello di maturit`a:  3\\ Valutazione attributi:  L\end{tabular} &
      \begin{tabular}[c]{@{}c@{}}Livello di maturit`a:  3\\ Valutazione attributi:  L\end{tabular} \\
    \multicolumn{1}{l}{Descrizione:} &
      \multicolumn{3}{l}{\begin{tabular}[c]{@{}l@{}}Seguendo le valutazioni dello standard ISO/IEC 15504, CoffeeCode\\ intende raggiungere un livello di maturità dei processi pari a 3 e per\\ la valutazione di ogni attributo un livello pari a L ossia che gli attributi\\ sono stati soddisfatti almeno per il 75\% della loro interezza.\end{tabular}} \\ \hline
    \begin{tabular}[c]{@{}c@{}}QoPR004 Rispetto dei ruoli \\ e identificazione nei prodotti\end{tabular} &
      \begin{tabular}[c]{@{}c@{}}MoPR004 Aderenza\\ ai ruoli\end{tabular} &
      0 &
      0 \\
    \multicolumn{1}{l}{Descrizione:} &
      \multicolumn{3}{l}{\begin{tabular}[c]{@{}l@{}}Le suddivisioni dei ruoli precedentemente definite, devono essere fisse\\ per ogni prodotto, per questo CoffeeCode prevede di avere 0 prodotti\\ nei quali i vari ruoli non sono stati rispettati.\end{tabular}} \\ \hline
    \begin{tabular}[c]{@{}c@{}}QoPR005 Rispetto del \\ versionamento dei prodotti\end{tabular} &
      \begin{tabular}[c]{@{}c@{}}MoPR005 Controllo\\ prodotti\end{tabular} &
      15+ &
      20+ \\
    \multicolumn{1}{l}{Descrizione:} &
      \multicolumn{3}{l}{\begin{tabular}[c]{@{}l@{}}Visionare il numero di commit eseguiti sulla repository è un ottimo modo\\ per tenere traccia delle modifiche avvenute sui prodotti, in modo da ricavare\\ facilmente l’ultima versione di ogni prodotto.Si è quindi deciso che il numero\\ di commit da effettuare in media durante una settimana produttiva è di minimo\\ 15 per l’accettazione dell’obiettivo, mentre per una maggiore qualità del processo\\ intendiamo effettuare almeno 20 commit settimanali.\end{tabular}} \\ \hline
    \caption{Obiettivi e metriche di qualità per il PRC001}
    \label{tab:my-table}\\
    \end{longtable}

    \begin{center}
        \centering
        \textbf{PRC002 Analisi}
    \end{center}
    \begin{longtable}[c]{cccc}
        \hline
        Obiettivo &
          Metrica &
          Valore accettabile &
          Valore desiderabili \\ \hline
        \endhead
        %
        \begin{tabular}[c]{@{}c@{}}QoPR006 Soddisfazione\\ dei requisiti obbligatori\end{tabular} &
          \begin{tabular}[c]{@{}c@{}}MoPR006 Verifica \\ requisiti obbligatori\end{tabular} &
          100\% &
          100\% \\
        \multicolumn{1}{l}{Descrizione:} &
          \multicolumn{3}{l}{\begin{tabular}[c]{@{}l@{}}I requisiti obbligatori, concordati con il cliente devono essere\\ tutti soddisfatti entro la data di fine progetto prevista.\end{tabular}} \\ \hline
        \begin{tabular}[c]{@{}c@{}}QoPR007 Soddisfazione dei requisiti \\ opzionali e desiderabili\end{tabular} &
          \begin{tabular}[c]{@{}c@{}}MoPR007 Verifica requisiti \\ opzionali\\ MoPR008 Verifica requisiti \\ desiderabili\end{tabular} &
          30\% &
          50\% \\
        \multicolumn{1}{l}{Descrizione:} &
          \multicolumn{3}{l}{\begin{tabular}[c]{@{}l@{}}Pur non essendo obbligatori i requisiti opzionali e desiderabili \\ il gruppo si prefigge l'obiettivo di realizzare almeno il 30\% di tali \\ requisiti presentati, per aumentare la qualità dell'intero prodotto finale.\end{tabular}} \\ \hline
        QoPR008 Verifica dei rischi previsti &
          \begin{tabular}[c]{@{}c@{}}MoPR009 Verifica rischi non \\ pervenuti\end{tabular} &
          0 &
          0 \\
        \multicolumn{1}{l}{Descrizione:} &
          \multicolumn{3}{l}{\begin{tabular}[c]{@{}l@{}}Durante il processo è probabile il verificarsi di imprevisti, questi devono\\ essere opportunamente rilevati già in fase di analisi. Si prevede quindi di\\ non avere ulteriori imprevisti rispetto a quelli preventivati\end{tabular}} \\ \hline
        \caption{Obiettivi e metriche di qualità per il PRC002}
        \label{tab:my-table}\\
        \end{longtable}

        \begin{center}
            \centering
            \textbf{PRC003 Produzione dei documenti}
        \end{center}
        \begin{longtable}[c]{cccc}
            \hline
            Obiettivo &
              Metrica &
              Valore accettabile &
              Valore desiderabili \\ \hline
            \endhead
            %
            \begin{tabular}[c]{@{}c@{}}QoPR09  Rispetto delle fasi del \\ ciclo di vita\end{tabular} &
              \begin{tabular}[c]{@{}c@{}}MoPR010 Analisi Way \\ of Working\end{tabular} &
              / &
              / \\
            \multicolumn{1}{l}{Descrizione:} &
              \multicolumn{3}{l}{\begin{tabular}[c]{@{}l@{}}Ogni documento attraversa determinate dasi del ciclo di vita del prodotto. \\ I verificatori hanno il compito di verificare la corretta attuazione del\\ way of working nel documento in ogni sua fase.\end{tabular}} \\ \hline
            \begin{tabular}[c]{@{}c@{}}QoPR010 Rispetto nella redazione \\ dei documenti\end{tabular} &
              MoPR011 Analisi documenti &
              3 &
              4+ \\
            \multicolumn{1}{l}{Descrizione:} &
              \multicolumn{3}{l}{\begin{tabular}[c]{@{}l@{}}Per mantenere un corretto aggiornamento e una corretta verifica delle \\ modifiche sui documenti, ogni documento dovrà essere visionato e \\ verificato almeno 3 volte da parte dei verificatori.\end{tabular}} \\ \hline
            \caption{Obiettivi e metriche di qualità per il PRC003}
            \label{tab:my-table}\\
            \end{longtable}

        \begin{center}
            \centering
            \textbf{PRC004 Verifica}
        \end{center}
        \begin{longtable}[c]{cccc}
            \hline
            Obiettivo &
              Metrica &
              Valore accettabile &
              Valore desiderabili \\ \hline
            \endhead
            %
            \begin{tabular}[c]{@{}c@{}}QoPR011 Attuare una verifica \\ costante\end{tabular} &
              \begin{tabular}[c]{@{}c@{}}MoPR012 Frequenza di \\ controlli\end{tabular} &
              / &
              / \\
            \multicolumn{1}{l}{Descrizione:} &
              \multicolumn{3}{l}{\begin{tabular}[c]{@{}l@{}}Il verificatore deve controllare il prodotto frequentemente, in modo\\  da correggerlo nella sua forma, contenuto e nelle funzionalità.\end{tabular}} \\ \hline
            \begin{tabular}[c]{@{}c@{}}QoPR012 Comunicare costantemente\\ durante la verifica\end{tabular} &
               &
              / &
              / \\
            \multicolumn{1}{l}{Descrizione:} &
              \multicolumn{3}{l}{\begin{tabular}[c]{@{}l@{}}Durante la realizzazione del progesso è necessario avere una\\ collaborazione e una comunicazione costante all'interno del\\ gruppo, in modo da evitare problemi di ogni genere e per\\ sincronizzarsi sul lavoro da svolgere.\end{tabular}} \\ \hline
            \begin{tabular}[c]{@{}c@{}}QoPR013 Rispettare le fasi \\ di verifica\end{tabular} &
              \begin{tabular}[c]{@{}c@{}}MoPR010 Analisi\\ Way of Working\end{tabular} &
              / &
              / \\
            \multicolumn{1}{l}{Descrizione:} &
              \multicolumn{3}{l}{\begin{tabular}[c]{@{}l@{}}Come per i documenti anche per ogni prodotto realizzato deve\\ essere sottoposto a verifiche da parte del Verificatore in modo\\ da controllare che i prodotti rispettino le norme indicate e per\\ disporre sempre di risultati verificati, attendibili ed analizzabili.\end{tabular}} \\ \hline
            \caption{Obiettivi e metriche di qualità per il PRC004}
            \label{tab:my-table}\\
            \end{longtable}
\end{document}