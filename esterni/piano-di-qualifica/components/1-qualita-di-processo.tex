\documentclass[../piano-di-qualifica.tex]{subfiles}

\begin{document}

\subsection{Scopo}%
\label{sub:scopo}
Lo scopo del documento è quello di ricerca la qualità di un prodotto che è soggetta a cambiamenti in base ai processi utilizzati nel corso della realizzazione dello stesso.
Si rende necessario quindi adottare un buon \glossario{CICLO DI VITA} che determina i processi da utilizzare che dovranno poi essere verificati e valutati attentamente per la corretta riuscita del prodotto.
A tal proposito si seguirà lo schema del \glossario{CICLO DI DEMING} per la gestione iterativa dei controlli e miglioramenti dei processi del prodotto.

\subsection{Nomenclatura metriche e obiettivi di qualità}%
\label{sub:nomenclatura_metriche_e_obiettivi_di_qualita}
Vengono di seguito spiegati gli obiettivi, le \glossario{METRICHE} e come essi verranno identificati, in modo sintetico:
\begin{itemize}
    \item \textbf{Obiettivi}: 
    \begin{center}
        \centering
        \textbf{QoPR[ID][Nome]}
    \end{center}
        \begin{itemize}
            \item \textbf{QoPR}: sta per "Quality of Process" ossia qualità del processo;
            \item \textbf{ID}: identificatore numerico a 3 cifre;
            \item \textbf{Nome}: riassunto della descrizione del processo.
        \end{itemize}
    \item \textbf{Metriche}: 
    \begin{center}
        \centering
        \textbf{MoPR[ID][Nome]}
    \end{center}
        \begin{itemize}
            \item \textbf{MoPR}: sta per "Metric of Process" ossia metrica del processo;
            \item \textbf{ID}: identificatore numerico a 3 cifre;
            \item \textbf{Nome}: riassunto della descrizione della metrica.
        \end{itemize}
\end{itemize}  
Ulteriori informazioni sulle metriche e sul loro funzionamento sono reperibili nel documento Norme di progetto.

\subsection{Processi}%
\label{sub:processi}
Come per metriche e obiettivi, i processi faranno affidamento alla seguente identificazione:
\begin{itemize}
    \item \textbf{Processi}: 
    \begin{center}
        \centering
        \textbf{PRC[ID][Nome]}
    \end{center}
    \begin{itemize}
        \item \textbf{PRC}: sta a indicare "Processo";
        \item \textbf{ID}: identificatore numerico a 3 cifre;
        \item \textbf{Nome}: riassunto della funzione del processo.
    \end{itemize}
\end{itemize}
Verranno poi fornite per ogni processo, come ulteriore arricchimento per le specifiche di ogni progetto:
\begin{itemize}
    \item Le funzioni principali da esso svolte;
    \item Gli obiettivi accettabili per garantire una qualità;
    \item Gli obiettivi desiderabili per ricercare una maggiore qualità;
    \item Le metriche adottate, nel caso ce ne fossero, per raggiungere tali obiettivi.
\end{itemize}

\subsubsection{PRC001 Organizzazione e pianificazione del progetto e della sua struttura}%
\label{sub:organizzazione_e_pianificazione_del_progetto_e_della_sua_struttura}
Questo \glossario{MACRO-PROCESSO} ha l'obiettivo di organizzare il lavoro da svolgere per la corretta soddisfazione dei \glossario{REQUISITI} identificati nel \glossario{PROGETTO}.
In questo processo viene messo in atto il \glossario{WAY OF WORKING}, redatto per la consultazione dell'intero \glossario{TEAM DI SVILUPPO}, il quale ha un'importanza significativa perché la corretta attuazione delle regole contenute al suo interno andranno a condizionare la qualità dell'intero progetto.
È quindi opportuno assicurarsi che la pianificazione temporale e i costi sostenuti siano supportati da metriche in grado di verificarne lo stato di avanzamento del progetto confrontandolo con la pianificazione precedentemente prodotta.

\textbf{Funzioni}

\begin{itemize}
    \item \textbf{Sviluppare \glossario{SOTTO-PROCESSI}}: ogni obiettivo prefissato deve essere associato ad azioni precise ognuna delle quali è raggruppata in un sotto-processo;
    \item \textbf{Dividere i compiti}: è fondamentale ,a corretta suddivisione dei compiti all'interno dei componenti del gruppo CoffeeCode al fine di non creare incomprensioni;
    \item \textbf{Calendarizzare i documenti}: definire delle \glossario{BASELINE} durante lo svolgimento del progetto;
    \item \textbf{Apprendimento autonomo}: data l'inesperienza del team nell'utilizzo di diverse tecnologie concordate nel \glossario{CAPITOLATO}, sarà richiesto un periodo di formazione personale più lungo. Questo periodo va conteggiato all'interno della calendarizzazione;
    \item \textbf{Standard appropriati}: vengono scelti gli standard più consoni alle finalità del progetto;
    \item \textbf{Stima budget}: bisogna essere a conoscenza delle proprie risorse in termini di costo/persona in modo da rispettare il \glossario{PREVENTIVO} stilato.
\end{itemize}

\textbf{Metriche}
%da realizzare

\textbf{Obiettivi}
\begin{itemize}
    \item \textbf{QoPR001 Rispetto delle scadenze della pianificazione}: l'obiettivo è rispettare al meglio le scadenza prefissate in modo da non avere rallentamenti durante la realizzazione del progetto;
    \item \textbf{QoPR002 Rispetto del budget istanziato}: devono potersi mantenere le risorse definite all'inizio del progetto;
    \item \textbf{QoPR003 Rispetto del ciclo di vita scelto}: ogni fase del Ciclo di Deming deve essere rispettata dai processi;
    \item \textbf{QoPR004 Rispetto dei ruoli e identificazione nei documenti}: deve essere possibile sapere i ruoli dei componenti nella partecipazione alla scrittura di un documento e le varie modifiche da loro fatte;
    \item \textbf{QoPR005 Rispetto del versionamento dei prodotti}: ogni prodotto realizzato durante i processi deve possedere un numero di versionamento in modo da tracciare le modifiche, e in caso di errori, trovare un punto di ripristino per il prodotto.
\end{itemize}

\subsubsection{PRC002 Analisi}
\label{sub:analisi}
In questo processo viene fatta un'analisi più generica, comprendente altri contenuti che non si riferiscono solo all'analisi dei requisiti (consultabile nel documento AnalisiDeiRequisiti) o all'analisi dei rischi (consultabile nel documento PianoDiProgetto), ma che comprende altre attività riportate dalle seguenti funzioni:

\textbf{Funzioni}
\begin{itemize}
    \item \textbf{Individuazione dei requisiti}: il testo del capitolato è presentato in modo più discorsivo che tecnico, vi è quindi la necessita di effettuare un'analisi per individuare requisiti: espliciti ,impliciti e classificarli in seguito;
    \item \textbf{Individuazione delle risorse}: analizzare le risorse e realizzare poi un preventivo è fondamentale, per il \glossario{FORNITORE} per verificare la fattibilità del progetto e per il clienti per valutare il costo del progetto;
    \item \textbf{Individuazione dei tempi}: capire la durata delle progetto e stabilire la scadenza delle varie fasi è importante per gestire le varie risorse senza incombere in perdite di tempo o in ritardi;
    \item \textbf{Individuazione dei rischi}: capire quali rischi possono presentarsi durante la realizzazione del progetto è fondamentale per prevedere e in caso attuare delle reazioni ad essi.
\end{itemize}

\textbf{Metriche}
%da realizzare

\textbf{Obiettivi}
\begin{itemize}
    \item \textbf{QoPR006 Soddisfazione dei requisiti obbligatori}: tutti i requisiti obbligatori, concordati con il cliente, devono essere soddisfatti a fine progetto;
    \item \textbf{QoPR007 Soddisfazione dei requisiti opzionali e desiderabili}: i requisiti opzionali e desiderabili devono essere definiti in modo corretto, verificati insieme al cliente e possono essere soddisfatti solo se sono state avanzate risorse, in termini di tempo/persona, prima della fine del progetto;
    \item \textbf{QoPR008 Verifica dei tempi previsti}: i tempi previsti e concordati devono essere rispettati;
    \item \textbf{QoPR009 Verifica dei rischi previsti}: per mantenere costante lo sviluppo del progetto non dovrebbero presentarsi rischi non preventivati che potrebbero comportare a ritardi sui tempi e a spreco di risorse.
\end{itemize}

\subsubsection{PRC003 Produzione dei documenti}
\label{sub:produzione_dei_documenti}
Questo processo rimane operativo per tutta la durata del progetto in quanto ha il compito di redigere i vari documenti, necessari nelle diverse fasi di sviluppo, che comprendano le scelte effettuate, gli strumenti utilizzati e le modifiche decise ed inserite durante la durata del progetto.

\textbf{Funzioni}
Il processo segue di pari passo il ciclo di vita di ogni documento, come spiegato nelle NormeDiProgetto e prevede le fasi di:
\begin{itemize}
    \item Redazione;
    \item Modifica;
    \item Verifica;
    \item Approvazione.        
\end{itemize}

\textbf{Obiettivi}
\begin{itemize}
    \item \textbf{QoPR010 Rispetto delle fasi del ciclo di vita}: deve essere rispettato ogni fase del ciclo di vita del documento e le scadenze previste;
    \item \textbf{QoPR011 Rispetto nella redazione dei documenti}: devono essere rispettate le linee guida inserite nelle NormeDiProgetto.
\end{itemize}

\subsubsection{PRC004 Verifica}
\label{sub:produzione_dei_documenti}
Durante tutto lo svolgimento del progetto, il processo di verifica sarà attivo con l'obiettivo principale quello di valutare la correttezza dei prodotti generati per: 
\begin{itemize}
    \item Individuare eventuali errori;
    \item Decidere se rispettano gli standard di qualità prefissati.
\end{itemize}

\textbf{Funzioni}
\begin{itemize}
    \item \textbf{Verificare i requisiti dei prodotti}: viene verificato che i prodotti soddisfino i requisiti richiesti concordati;
    \item \textbf{Verificare l'aderenza alle Norme di Progetto}: viene verificato il rispetto del Way Of Working, contenuto nelle Norme di Progetto.
\end{itemize}

\textbf{Metriche}
%da realizzare

\textbf{Obiettivi}
\begin{itemize}
    \item \textbf{QoPR012 Attuare una verifica costante}: si effettuano verifiche costanti in ogni fase del progetto, testando e controllando il prodotto ogni volta che viene modificato o che si è in procinto di terminare una \glossario{MILESTONE};
    \item \textbf{QoPR013 Comunicare costantemente durante la verifica}: mantenere una comunicazione costante con tutti i ruoli all'interno del gruppo;
    \item \textbf{QoPR014 Rispettare le fasi di verifica}: si esegue la fase di verifica in modo iterativo, rispettando sempre ciò che è stato deciso all'interno delle Norme di Progetto, in modo da rendere coerenti i risultati ottenuti.
\end{itemize}

\subsection{Tabelle di qualità di processo}
\label{sub:tabelle_di_qualita_di_processo}
Gli obiettivi di qualità, discussi nelle precedenti sezioni, che devono essere parte integrante di ogni processo, verranno indicati in tabelle in questa sezione.
Per ogni obiettivo viene indicato:

\begin{itemize}
   \item \textbf{Obiettivo}: indica il codice identificativo dell'obiettivo come descritto nella sezione §2.2;
   \item \textbf{Metrica}: indica, se presente, la metrica adottata per la valutazione dell'obiettivo di qualità come descritto nella sezione §2.2;
   \item \textbf{Valore accettabile}: rappresenta il valore minimo di qualità dell'obiettivo che CoffeCode intende ottenere. Non è presente in caso di mancanza della metrica associata all'obiettivo;
   \item \textbf{Valore desiderato}: rappresenta il valore di qualità dell'obiettivo che CoffeCode intende ottenere una maggiore qualità rispetto a quella minima. Non è presente in caso di mancanza della metrica associata all'obiettivo;
   \item \textbf{Descrizione}: descrizione generale dell'obiettivo.
\end{itemize}
%Tabelle

\begin{center}
    \centering
    \textbf{PRC001 Organizzazione e pianificazione del progetto e della sua struttura}
\end{center}
    \begin{table}[H]
        \centering
        \begin{tabular}{cccc}
        Obiettivo    & Metrica & Valore accettabile & Valore desiderabile \\
                     &         &                    &                    \\
        Descrizione: & \multicolumn{3}{c}{}        
        \end{tabular}
        \caption{Obiettivi e metriche di qualità per il PRC001}
    \end{table}

    \begin{center}
        \centering
        \textbf{PRC002 Analisi}
    \end{center}
        \begin{table}[H]
            \centering
            \begin{tabular}{cccc}
            Obiettivo    & Metrica & Valore accettabile & Valore desiderabile \\
                         &         &                    &                    \\
            Descrizione: & \multicolumn{3}{c}{}        
            \end{tabular}
            \caption{Obiettivi e metriche di qualità per il PRC002}
        \end{table}

        \begin{center}
            \centering
            \textbf{PRC003 Produzione dei documenti}
        \end{center}
            \begin{table}[H]
                \centering
                \begin{tabular}{cccc}
                Obiettivo    & Metrica & Valore accettabile & Valore desiderabile \\
                             &         &                    &                    \\
                Descrizione: & \multicolumn{3}{c}{}        
                \end{tabular}
                \caption{Obiettivi e metriche di qualità per il PRC003}
            \end{table}

        \begin{center}
            \centering
            \textbf{PRC004 Verifica}
        \end{center}
            \begin{table}[H]
                \centering
                \begin{tabular}{cccc}
                Obiettivo    & Metrica & Valore accettabile & Valore desiderabile \\
                             &         &                    &                    \\
                Descrizione: & \multicolumn{3}{c}{}        
                \end{tabular}
                \caption{Obiettivi e metriche di qualità per il PRC004}
            \end{table}
\end{document}