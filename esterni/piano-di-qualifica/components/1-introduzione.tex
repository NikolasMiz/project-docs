\documentclass[../piano-di-qualifica.tex]{subfiles}
\begin{document}

\subsection{Premessa}%
\label{sub:premessa}
Il piano di qualifica è un documento soggetto a continue modifiche nel corso dell'intera durata del \glossario{progetto}.
Tutto ciò è comprensibile dato che le \glossario{metriche} e i \glossario{processi}, identificati inizialmente, possono rivelarsi insufficienti o non adatti al fine di mantenere un'ottima qualità del progetto e all'interno del team di lavoro.
Per queste ragioni il documento sarà prodotto incrementalmente nel corso dello svolgimento del progetto.

\subsection{Scopo del documento}%
\label{sub:scopo_del_documento}
Lo scopo del documento è quello di illustrare formalmente tutte le modalità di verifica, validazione e le norme di qualità adottate, con il fine ultimo quello di preservare la qualità di prodotto e di processo.
Per raggiungere l'obiettivo prefissato verranno svolte continuamente verifiche sulle attività e sui processi in corso in modo da identificare nel minor tempo possibili errori e/o anomalie in modo da rendere più facile gli interventi di manutenzione e riducendo al minimo lo spreco di risorse.

\subsection{Scopo del prodotto}%
\label{sub:scopo_del_prodotto}
Lo scopo del prodotto è la realizzazione di due \glossario{plug-in}, in \glossario{Javascript}, per ottenere delle elaborazioni su il flusso di dati raccolti dai sistemi dell’azienda,
per poter poi prevedere i loro stati futuri così da inviare possibile segnalazioni o allarmi agli operatori del servizio cloud e alla linea di produzione del software.
In particolare le previsioni che si desiderano ottenere sono di due tipi:
\begin{itemize}
    \item ”Classificazioni”: per valutare il gruppo di appartenenza degli eventi dei dati ”\glossario{predittori}”;
    \item ”Regressioni”: nel caso in cui il valore cercato sia numerico e con campo continuo. 
\end{itemize}
I due plug-in devono estendere lo strumento di monitoraggio dei dati in \glossario{Grafana} applicando il \glossario{Machine Learning} attraverso le tecniche di 
\glossario{Support Vector Machine} e di \glossario{Regressione Lineare} utilizzate per la predizione su dati.

\subsection{Glossario}%
\label{sub:glossario}
All'interno del documento sono presenti termini che possono avere dei significati ambigui a seconda del contesto. Per evitare questa ambiguità è stato creato un documento di nome \textit{Glossario} che conterrà tali termini con il loro significato specifico. Per segnalare che il termine del testo è presente all'interno del glossario verrà segnalato con una sottolineatura e una G a pedice a fianco del termine.

\subsection{Riferimenti}%
\label{sub:riferimenti}

\subsubsection{Normativi}%
\label{subs:normativi}
\begin{itemize}
    \item \textbf{Norme di progetto}: \textit{Norme di progetto};
    \item \textbf{Capitolato d'appalto C4}: \href{https://www.math.unipd.it/~tullio/IS-1/2019/Progetto/C4.pdf}{Predire in Grafana}.
\end{itemize}

\subsubsection{Informativi}%
\label{subs:informativi}
\begin{itemize}
    \item \textbf{Presentazione Capitolato C4}: \href{https://www.math.unipd.it/~tullio/IS-1/2019/Dispense/C4a.pdf}{Predire in Grafana};
    \item \textbf{Slide del corso di Ingegneria del Software (SWE)}:
    \begin{itemize}
        \item Qualità di prodotto: \href{https://www.math.unipd.it/~tullio/IS-1/2019/Dispense/L12.pdf}{https://www.math.unipd.it/~tullio/IS-1/2019/Dispense/L12.pdf};
        \item Qualità di processo: \href{https://www.math.unipd.it/~tullio/IS-1/2019/Dispense/L13.pdf}{https://www.math.unipd.it/~tullio/IS-1/2019/Dispense/L13.pdf};
        \item Verifica e validazione: \href{https://www.math.unipd.it/~tullio/IS-1/2019/Dispense/L14.pdf}{https://www.math.unipd.it/~tullio/IS-1/2019/Dispense/L14.pdf}.
    \end{itemize}
    \item \textbf{Standard ISO/IEC 15504 (SPYCE)}: \href{https://www.cs.helsinki.fi/u/paakki/Pyhajarvi.pdf}{https://www.cs.helsinki.fi/u/paakki/Pyhajarvi.pdf};
    \item \textbf{Standard ISO/IEC 9126}: \href{https://it.wikipedia.org/wiki/ISO/IEC\_9126}{https://it.wikipedia.org/wiki/ISO/IEC\_9126};
    \item \textbf{Ciclo di Deming}: \href{https://it.wikipedia.org/wiki/Ciclo\_di\_Deming}{https://it.wikipedia.org/wiki/Ciclo\_di\_Deming}.
\end{itemize}

\end{document}