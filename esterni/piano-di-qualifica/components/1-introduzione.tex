\documentclass[../piano-di-qualifica.tex]{subfiles}
\begin{document}

\subsection{Premessa}%
\label{sub:premessa}
Il piano di qualifica è un documento soggetto a continue modifiche nel corso dell'intera durata del \glossario{progetto}.
Tutto ciò è comprensibile dato che le \glossario{metriche} e i \glossario{processi}, identificati inizialmente, possono rivelarsi insufficienti o non adatti al fine di mantenere un'ottima qualità del progetto e all'interno del team di lavoro.
Per queste ragioni il documento sarà prodotto incrementalmente nel corso dello svolgimento del progetto.

\subsection{Scopo del documento}%
\label{sub:scopo_del_documento}
Lo scopo del documento è quello di illustrare formalmente tutte le modalità di verifica, validazione e le norme di qualità adottate, con il fine ultimo quello di preservare la qualità di prodotto e di processo.
Per raggiungere l'obiettivo prefissato verranno svolte continuamente verifiche sulle attività e sui processi in corso in modo da identificare nel minor tempo possibili errori e/o anomalie in modo da rendere più facili gli interventi di manutenzione e riducendo al minimo lo spreco di risorse.

\subsection{Scopo del prodotto}%
\label{sub:scopo_del_prodotto}
Lo scopo del prodotto è la realizzazione di un \glossario{plug-in}, scritto in linguaggio \glossario{Javascript}, per lo strumento \glossario{open source} \glossario{Grafana},
che applichi le tecniche di \glossario{Machine Learning} della \glossario{Regressione Lineare} e del \glossario{Support Vector Machine} a un flusso di dati.
Questo plug-in provvederà quindi a svolgere un'analisi sui dati e da questi, fornire una predizione al fine di monitorare la \glossario{liveliness} del sistema e di consigliare 
interventi alla linea di produzione del software tramite specifici \glossario{alert}.
Prima della realizzazione di tale plug-in, si dovrà sviluppare un programma comprendente un'interfaccia web, attraverso la quale verrà eseguito l'\glossario{addestramento} necessario al corretto
funzionamento degli algoritmi di Machine Learning.
I risultati ottenuti dovranno poi essere visualizzati, in forma di grafico, in una \glossario{dashboard} sulla piattaforma Grafana.

\subsection{Glossario}%
\label{sub:glossario}
All'interno del documento sono presenti termini che possono avere dei significati ambigui a seconda del contesto. Per evitare questa ambiguità è stato creato un documento di nome \textit{Glossario v1.0.0}, che conterrà tali termini con il loro significato specifico. Per segnalare che il termine del testo è presente all'interno del glossario verrà segnalato con una sottolineatura e una G maiuscola come pedice a fianco del termine.

\subsection{Riferimenti}%
\label{sub:riferimenti}

\subsubsection{Normativi}%
\label{subs:normativi}
\begin{itemize}
    \item \textbf{Norme di progetto}: \textit{Norme di progetto};
    \item \textbf{Capitolato d'appalto C4}: \href{https://www.math.unipd.it/~tullio/IS-1/2019/Progetto/C4.pdf}{Predire in Grafana}.
\end{itemize}

\subsubsection{Informativi}%
\label{subs:informativi}
\begin{itemize}
    \item \textbf{Presentazione Capitolato C4}: \href{https://www.math.unipd.it/~tullio/IS-1/2019/Dispense/C4a.pdf}{Predire in Grafana};
    \item \textbf{Slide del corso di Ingegneria del Software (SWE)}:
    \begin{itemize}
        \item Qualità di prodotto: \href{https://www.math.unipd.it/~tullio/IS-1/2019/Dispense/L12.pdf}{https://www.math.unipd.it/~tullio/IS-1/2019/Dispense/L12.pdf};
        \item Qualità di processo: \href{https://www.math.unipd.it/~tullio/IS-1/2019/Dispense/L13.pdf}{https://www.math.unipd.it/~tullio/IS-1/2019/Dispense/L13.pdf};
        \item Verifica e validazione: \href{https://www.math.unipd.it/~tullio/IS-1/2019/Dispense/L14.pdf}{https://www.math.unipd.it/~tullio/IS-1/2019/Dispense/L14.pdf}.
    \end{itemize}
    \item \textbf{Standard ISO/IEC 15504 (SPYCE)}: \href{https://www.cs.helsinki.fi/u/paakki/Pyhajarvi.pdf}{https://www.cs.helsinki.fi/u/paakki/Pyhajarvi.pdf};
    \item \textbf{Standard ISO/IEC 9126}: \href{https://it.wikipedia.org/wiki/ISO/IEC\_9126}{https://it.wikipedia.org/wiki/ISO/IEC\_9126};
    \item \textbf{Ciclo di Deming}: \href{https://it.wikipedia.org/wiki/Ciclo\_di\_Deming}{https://it.wikipedia.org/wiki/Ciclo\_di\_Deming}.
\end{itemize}

\end{document}