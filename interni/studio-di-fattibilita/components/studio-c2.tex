\documentclass[../studio-di-fattibilita.tex]{subfiles}

\begin{document}

\subsection{Informazioni generali}%
\label{sub:informazioni_generale}
\begin{description}
  \item Nome: Etherless
  \item Proponente: Red Babel
  \item Committente: prof. Tullio Vardanega e prof. Riccardo Cardin
\end{description}

\subsection{Descrizione}%
\label{sub:descrizione}
\textit{Etherless} è una piattaforma \glossario{cloud} che permette agli sviluppatori di caricare delle funzioni Javascript e consente
agli utenti finali di pagare per la loro esecuzione. Si basa sul concetto di \textit{Computation-as-a-Service} (o \textit{CaaS}).
Una parte di questo pagamento viene trattenuta dalla piattaforma come pagamento per il servizio.

\subsection{Finalità del progetto}%
\label{sub:finalita_del_progetto}
Il progetto consiste nel creare una piattaforma \glossario{cloud} composta nella seguente maniera:
\begin{itemize}
  \item \textit{etherless-cli}: è il modulo che permette agli \textit{sviluppatori} di interagire con la piattaforma.
  Supporta i vari comandi utili agli \textit{sviluppatori} quali:
  \begin{itemize}
    \item Configurare conto Ethereum
    \item Caricare codice eseguibile
    \item Visualizzare la lista delle funzioni caricate
    \item Eseguire una funzione e riceverne il valore di return
    \item Vedere i log delle esecuzioni
  \end{itemize}
  \item \textit{etherless-smart}: è un set di \textit{Smart Contracts} che permettono di gestire la comunicazione tra 
        \textit{etherless-cli} e \textit{etherless-server}, così come il pagamento del servizio in ETH (Ethereum).
  \item \textit{etherless-server}: il modulo che esegue le funzioni e attraverso \textit{etherless-smart} comunica
        con \textit{etherless-cli}. Quando la funzione restituisce un valore di ritorno oppure incontra un'eccezione
        emette un evento nella Blockchain, tale evento verrà ricevuto da \textit{etherless-cli} che mostrerà il risultato
        allo \textit{sviluppatore}.
\end{itemize}

Inoltre il progetto richiede i seguenti vincoli:
\begin{itemize}
      \item \textit{etherless-cli} deve essere installato utilizzando il node package manager (\textit{npm}) utilizzando
            il comando 
            
            \texttt{npm install -g etherless-cli}.
      \item Una volta installato lo sviluppatore deve essere in grado di eseguire i comandi per fare le seguenti
            azioni: inizializzare ed accedere ad un conto \textit{Ethereum}, caricare, eliminare ed eseguire una funzione,
            visualizzare la lista delle funzioni caricate.
      \item Gli \textit{Smart Contracts} devono poter essere ampliati.
      \item \textit{Etherless} deve essere sviluppato utilizzando \glossario{Typescript} 3.6.
      \item Devono essere utilizzati \textit{typescript-eslint} ed \glossario{ESLint} durante il processo.
      \item \textit{etherless-server} deve essere implementato utilizzando un \glossario{Framework} \glossario{Serverless}.
\end{itemize}


\subsection{Tecnologie interessate}%
\label{sub:tecnologie_interessate}
\begin{itemize}
  \item \textbf{Ethereum} è una piattaforma che permette agli utenti di scriviere agevolmente applicazioni decentralizzate
        (ĐApps) utilizzando la Blockchain.
  \item \textbf{Blockchain} è un database condiviso che tiene traccia di tutti i movimenti di criptovaluta utilizzando
        una struttura a blocchi. Una volta che un blocco è stato verificato esso viene inserito irreversibilmente nella 
        Blockchain.
  \item \textbf{EVM (Ethereum Virtual Machine)} è una macchina virtuale decentralizzata che permette di eseguire script.
  \item \textbf{Smart Contracts}, sono codice che viene eseguito dalla \textbf{EVM}, può contenere \textit{ETH}, dati o
        una combinazione di essi.
  \item \textbf{Solidity} è un linguaggio utilizzato per scrivere gli \textit{Smart Contracts}.
  \item \textbf{AWS Lambda} è una piattaforma di calcolo \textit{\glossario{Serverless}} guidata dagli eventi. Può eseguire del codice 
        in risposta a degli eventi.
  \item \textbf{TypeScript} è un linguaggio che estende le potenzialità di Javascript aggiungendo o rendendo più
        flessibili le sue caratteristiche.

        Viene consigliata la versione 3.6.
\end{itemize}

\subsection{Aspetti positivi}%
\label{sub:aspetti_positivi}
Il capitolato risulta interessante perchè utilizza tecnologie relativamente recenti quali la \glossario{Blockchain}
e il \glossario{Framework} \glossario{Serverless} non trattate dal nostro piano di studi.  

\subsection{Rischi}%
\label{sub:rischi}
Il gruppo non ha conoscenze approfondite delle tecnologie necessarie per lo sviluppo del capitolato. Richiederebbe 
quindi uno studio approfondito che, visti in tempi ristretti potrebbe inficiare sulla parte di sviluppo.

\subsection{Conclusione}%
\label{sub:conclusione}
Dopo aver valutato il capitolato il gruppo ha deciso di non considerarlo come prima scelta. 

\end{document}
