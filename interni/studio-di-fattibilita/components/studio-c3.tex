\documentclass[../studio-di-fattibilita.tex]{subfiles}

\begin{document}

\subsection{Informazioni generali}%
\label{sub:informazioni_generale}
\begin{description}
  \item Nome: NaturalAPI
  \item Proponente: teal.blue
  \item Committente: prof. Tullio Vardanega e prof. Riccardo Cardin
\end{description}

\subsection{Descrizione}%
\label{descrizione}
Le difficoltà comunicative tra i diversi \glossario{Stakeholder} è uno dei principali problemi durante la realizzazione di un progetto software, provocando frantendimenti e ritardi che di conseguenza aumenta i costi del progetto. Una possibile soluzione può essere dare ai sviluppatori il compito di descrivere le proprie soluzioni mediante il codice che tuttavia sarà difficile da capire per il cliente. Gherkin è lo standard che permette di descrivere scenari in linguaggio naturale secondo un modello standard che viene poi collegato al codice dell'applicazione. Il \glossario{Behaviour-Driven Development (BDD)} ha come scopo l'utilizzo del linguaggio naturale per chiarire la comunicazione tra i vari stakeholders usando modelli di linguaggio standardizzati privando così di infraintendimenti. Però il linguaggio naturale e il codice rimancono ancora collegati solo dall'intervento dello sviluppatore, che deve stabilire la firma del metodo dell'API a cui vengono passati come parametro lo scenario scritto in linguaggio naturale
\subsection{Finalità del progetto}%
\label{finalita_del_progetto}
L'obiettivo è quello di creare un toolkit \glossario{Proof-of-Concept} chiamato NaturalAPI che permette di ridurre il divario tra le specifiche del progetto in linguaggio naturale e le API. Questa finalità permette agli sviluppatori futuri di scrivere API più gestibili,prevedibili e coerenti partendo fin dai requisiti del progetto. NaturalAPI dovrà quindi essere composto di 3 elementi: 
\begin{itemize}
  \item NaturalAPI Discover: un \glossario{Business Domain Language (BDL)} extractor, in grado di indivuare i termini più usati dentro uno specifico tipo di business e le relazioni tra di loro (nomi, predicati, verbi);
  \item NaturalAPI Develop: un esportatore di linguaggio che converte il \glossario{BAL} in test e API in uno dei linguaggio e \glossario{Framework} disponibili all'applicazione supportando la creazioni di nuove \glossario{Repository} di codice e l'aggiornamento di quelle esistenti 
  \item NaturalAPI Design: un parser di scenari e funzionalità che crei un'API \glossario{Business Application Language (BAL)} a partire dai documenti Gherkin e un Business Domain Language (BDL) 
\end{itemize}

\subsection{Tecnologie interessate}%
\label{tecnologie_interessate}
\begin{itemize}
  \item Natural language processing
  \item Gherkin: standard per produrre user case in formato \glossario{BDD}
  \item OpenApi: formato per rappresentare \glossario{API}
  \item Cucumberm o Heaptest: strumenti software per supportare il \glossario{BDD}
  \item Swagger: software \glossario{Framework} per creare \glossario{API} basato sul formato OpenAPI
  \item Standard Parser: parser open source
  \item Behavioural Driven Development
\end{itemize}

\subsection{Aspetti positivi}%
\label{aspetti_positivi}
Tale progetto è particolarmente interesante per le proprie finalità. La comunicazioni infatti è uno dei aspetti più importanti durante un progetto ed avere uno strumento per standardizzare il linguaggio naturale usato dai Stakeholders provocherebbe sicuramente un incremento in termini di efficacia. Questo capitolato risulta stimulante per le finalità del progetto proposto e permetterebbe di approfondire nuove tecnologie interessanti quali le BDD. 

\subsection{Rischi}%
\label{rischi}
Il più grande rischio riscontrato il capitolato è la mancata conoscenza dei componenti del team delle tecnologie di Natural language processisng, \glossario{BDD} e di conseguenza Gherkin e Cucumber; provocando di conseguenza un necessario studio preliminare affinchè il team abbia la possibilità di sfruttare i concetti richiesti nel capitolato mettendoli in pratica

\subsection{Conclusione}%
\label{sub:conclusione}
Il capitolato non è stato preso in considerazione perchè concordato dal team che lo studio premilinare delle tecnologie e concetti risulta di impatto troppo alto sui tempi del progetto, decidendo quindi di non proporsi per il capitolato.

\end{document}
