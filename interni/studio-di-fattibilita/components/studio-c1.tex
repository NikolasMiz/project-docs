\documentclass[../studio-di-fattibilita.tex]{subfiles}

\begin{document}

\subsection{Informazioni generali}%
\label{sub:informazioni_generale}
\begin{description}
  \item Nome: Autonomous Highlights Platform
  \item Proponente: Zero12
  \item Committente: prof. Tullio Vardanega e prof. Riccardo Cardin
\end{description}

\subsection{Descrizione}%
\label{sub:descrizione}
Il capitolato C1 prevede la creazione di una piattaforma web che mediante l'inserimento di un video relativo ad un evento sportivo identificato, riesca a produrre un ulteriore video della durata massima di 5 minuti, contenente i momenti salienti (\glossario{highlights}) del video precedentemente inserito.

\subsection{Finalità del progetto}%
\label{sub:finalita_del_progetto}
Il \glossario{progetto} consiste nel creare una piattaforma web dotata di un sistema di \glossario{Machine Learning} in grado di identificare ogni momento importante di un evento.
Le principali funzionalità che la piattaforma deve offrire sono:
\begin{itemize}
  \item Caricamento del video scelto;
  \item Identificazione dei momenti salienti all'interno del video;
  \item Estrazione delle corrispondenti parti di video;
  \item Generazione della sintesi del video;
\end{itemize}
Mentre i vincoli obbligatori del progetto sono:
\begin{itemize}
  \item L'utilizzo di \glossario{Sage Maker};
  \item Realizzare l'architettura basata a \glossario{micro-servizi}, in modo da dividere il progetto in diverse funzionalità ognuna indipendente dalle altre;
  \item Caricamento dei video da elaborare tramite riga di comando;
  \item Realizzazione di una console web per l'analisi e il controllo degli stati di elaborazione del video;
\end{itemize}


\subsection{Tecnologie interessate}%
\label{sub:tecnologie_interessate}
Per il progetto viene consigliato di usare la tecnologia di \glossario{Amazon Web Services} e in particolare i seguenti servizi:
\begin{itemize}
  \item Elastic  Container  Service  o  Elastic  Kubernetes  Service: servizio  di  orchestrazione  di  contenitori  altamente  dimensionabile  ad  elevate prestazioni;
  \item \glossario{DynamoDB}: database NoSQL dalle alte performance ideale per la conservazione di tag o altre informazioni a supporto dell'applicativo;
  \item \glossario{AWS Transcode}: servizio gestito per la conversione ed elaborazione di diversi formati video;
  \item Sage Maker: servizio completamente gestito che copre l'intero flusso di lavoro dell'apprendimento automatico per;
  \begin{itemize}
    \item Etichettare e preparare i dati;
    \item Scegliere un algoritmo;
    \item Formare il \glossario{modello} e ottimizzarlo per la distribuzione;
    \item Effettuare previsioni;
    \item Intraprendere azioni;
  \end{itemize} 
  \item AWS Rekognition video: servizio di analisi video basato su apprendimento approfondito, in grado di:
  \begin{itemize}
    \item Riconoscere i movimenti delle persone in un fotogramma;
    \item Riconoscere soggetti, volti, oggetti, celebrità e contenuti inappropriati;
  \end{itemize} 
\end{itemize}
Come linguaggi di programmazione vengono consigliati:
\begin{itemize}
    \item \glossario{NodeJS}: per lo sviluppo di API Restful JSON a supporto dell’applicativo;
    \item \glossario{Python}: per lo sviluppo delle componenti di Machine Learning;
\end{itemize}
Per la realizzazione dell'interfaccia web vengono consigliati:
\begin{itemize}
  \item \glossario{HTML5};
  \item \glossario{CSS3};
  \item \glossario{Javascript}, utilizzando un \glossario{framework} responsive come \glossario{Twitter};
\end{itemize}

\subsection{Aspetti positivi}%
\label{sub:aspetti_positivi}
Il progetto presenta adeguatamente l'idea generale del software da realizzare, fornendo le informazioni e la documentazione necessaria per 
comprendere ogni singola tecnologia di sviluppo da utilizzare e specificando i documenti e i prodotti richiesti dall'azienda nelle fasi di pre-implementazione e post-implementazione.
Si presenta inoltre come un buon sistema per velocizzare l'acquisizione dei momenti salienti di un evento sportivo appena concluso.

\subsection{Rischi}%
\label{sub:rischi}
L’impiego di numerose tecnologie richiede un impegno non indifferente che potrebbe comportare a ritardi nella consegna del software e a una non corretta implementazione delle funzionalità richieste, oltre al fatto che non tutte le funzionalità della piattaforma vengono chiaramente esplicitate.

\subsection{Conclusione}%
\label{sub:conclusione}
Il capitolato non è stato preso in considerazione per la non disponibilità a ulteriori gruppi di lavoro da parte del proponente.

\end{document}
