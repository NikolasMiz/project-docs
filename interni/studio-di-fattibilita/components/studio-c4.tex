\documentclass[../studio-di-fattibilita.tex]{subfiles}

\begin{document}

\subsection{Informazioni generali}%
\label{sub:informazioni_generale}
\begin{description}
  \item \textbf{Nome}: Predire in Grafana.
  \item \textbf{Proponente}: Zucchetti S.p.a..
  \item \textbf{Committente}: prof. Tullio Vardanega e prof. Riccardo Cardin.
\end{description}

\subsection{Descrizione}%
\label{sub:descrizione}
Zucchetti chiede lo sviluppo di un software che applichi il \glossario{machine learning} al flusso dei dati raccolti dai sistemi dell’azienda per poter prevedere i loro stati futuri così da poter intervenire in tempo se necessario.

\subsection{Finalità del progetto}%
\label{sub:finalita_del_progetto}
Il progetto ha come obiettivo la realizzazione di un \glossario{plug-in} per lo strumento di monitoraggio di \glossario{Grafana}.
Il software dovrà soddisfare almeno i seguenti requisiti:
\begin{itemize}
  \item Produrre un file JSON dai dati di addestramento con i \glossario{parametri} per le previsioni con \glossario{SVM} per le classificazioni o la \glossario{Regressione Lineare};
  \item Leggere la definizione del \glossario{predittore} dal file in formato \glossario{JSON};
  \item Associare i predittori letti dal file JSON al flusso di dati presente in Grafana;
  \item Applicare la previsione e fornire i nuovi dati ottenuti al sistema di Grafana;
  \item Rendere disponibili i dati al sistema di creazione di grafici e \glossario{dashboard} per la loro visualizzazione.
\end{itemize}
Una volta soddisfatti i requisiti obbligatori si potranno implementare le seguenti caratteristiche:
\begin{itemize}
  \item Possibilità di definire \glossario{''alert''} in base a livelli di soglia raggiunti dai nodi collegati alle previsioni;
  \item Fornire i dati di bontà dei \glossario{modelli di previsione}. \glossario{''Precision''} e \glossario{''Recall''} per le SVM e R\textsuperscript{2} per la Regressione Lineare;
  \item Possibilità di applicare delle trasformazioni alle misure lette dal campo per ottenere delle \glossario{regressioni} esponenziali o logaritmiche e non solo lineari
  \item Possibilità di addestrare la SVM o la Regressione Lineare direttamente in Grafana;
  \item Implementare dei meccanismi di apprendimento di flusso, in modo da poter disporre di sistemi di previsione in costante adattamento ai dati rilevati sul campo;
  \item Utilizzare anche altri metodi di previsione, tra cui la versione delle SVM adattate alla regressione o piccole \glossario{Reti Neurali} per la classificazione.
\end{itemize}

\subsection{Tecnologie interessate}%
\label{sub:tecnologie_interessate}
\begin{itemize}
  \item Grafana: un software \glossario{Open Source} utilizzato per monitorare infrastrutture provvedendo grafici, diagrammi e allarmi;
  \item \glossario{JavaScript}: il linguaggio di programmazione richiesto per lo sviluppo del plug-in di Grafana;
  \item \glossario{Modelli di apprendimento}: SVM e Regressione Lineare
  \item \glossario{Java JMX}: uno strumento utilizzato per il monitoraggio di applicazioni sviluppate in \glossario{Java};
  \item \glossario{Java JMeter}: uno strumento utilizzato per verificare il comportamento di \glossario{applicazioni web}.
\end{itemize}

\subsection{Aspetti positivi}%
\label{sub:aspetti_positivi}
\begin{itemize}
  \item Le conoscenze di machine learning vengono sempre più richieste dalle imprese quindi sarebbe un’opportunità poterle imparare;
  \item L’azienda è ampiamente disponibile a fornire la formazione sugli algoritmi e tecnologie di machine learning.
\end{itemize}

\subsection{Rischi}%
\label{sub:rischi}
\begin{itemize}
  \item Il corso di studi della laurea triennale non copre il machine learning quindi si dovrà utilizzare una tecnologia a noi poco familiare.
\end{itemize}

\subsection{Conclusione}%
\label{sub:conclusione}
Il progetto ha riscontrato grande interesse nel gruppo e il poter essere formati da una grande \glossario{software house} come Zucchetti ci ha spinto ad optare per questo capitolato.

\end{document}
