\documentclass[../studio-di-fattibilita.tex]{subfiles}

\begin{document}

\subsection{Informazioni generali}%
\label{sub:informazioni_generale}
\begin{description}
  \item Nome: Predire in Grafana
  \item Proponente: Zucchetti Spa
  \item Committente: prof. Tullio Vardanega e prof. Riccardo Cardin
\end{description}

\subsection{Descrizione}%
\label{sub:descrizione}
Zucchetti chiede lo sviluppo di un software che applichi il \glossario{Machine Learning} al flusso dei dati raccolti dai sistemi dell’azienda per poter prevedere i loro stati futuri così da poter intervenire in tempo se necessario.

\subsection{Finalità del progetto}%
\label{sub:finalita_del_progetto}
Il progetto ha come obiettivo la realizzazione di un \glossario{plug-in} per lo strumento di monitoraggio di \glossario{Grafana}.
Il software dovrà soddisfare almeno i seguenti requisiti:
\begin{itemize}
  \item Produrre un file \glossario{JSON} dai dati di addestramento con i \glossario{parametri} per le previsioni con \glossario{SVM} per le classificazioni o la \glossario{Regressione Lineare} 
  \item Leggere la definizione del \glossario{predittore} dal file in formato \glossario{JSON}
  \item Associare i \glossario{predittori} letti dal file \glossario{JSON} al flusso di dati presente in \glossario{Grafana}
  \item Applicare la previsione e fornire i nuovi dati ottenuti al sistema di \glossario{Grafana}
  \item Rendere disponibili i dati al sistema di creazione di grafici e \glossario{dashboard} per la loro visualizzazione   
\end{itemize}
Una volta soddisfatti i requisiti obbligatori si potranno implementare le seguenti caratteristiche:
\begin{itemize}
  \item Possibilità di definire \glossario{''alert''} in base a livelli di soglia raggiunti dai nodi collegati alle previsioni
  \item Fornire i dati di bontà dei \glossario{modelli di previsione}. \glossario{''Precision''} e \glossario{''Recall''} per le \glossario{SVM} e \glossario{\(R^{2}\)} per la \glossario{Regressione Lineare}
  \item Possibilità di applicare delle trasformazioni alle misure lette dal campo per ottenere delle \glossario{regressioni} esponenziali o logaritmiche e non solo lineari
  \item Possibilità di addestrare la \glossario{SVM} o la \glossario{Regressione Lineare} direttamente in \glossario{Grafana}
  \item Implementare dei meccanismi di apprendimento di flusso, in modo da poter disporre di sistemi di previsione in costante adattamento ai dati rilevati sul campo
  \item Utilizzare anche altri metodi di previsione, tra cui la versione delle \glossario{SVM} adattate alla \glossario{regressione} o piccole \glossario{Reti Neurali} per la classificazione   
\end{itemize}

\subsection{Tecnologie interessate}%
\label{sub:tecnologie_interessate}
\begin{itemize}
  \item \glossario{Grafana}: un software \glossario{Open Source} utilizzato per monitorare infrastrutture provvedendo grafici, diagrammi e allarmi.
  \item \glossario{JavaScript}: il linguaggio di programmazione richiesto per lo sviluppo del \glossario{plug-in} di \glossario{Grafana}
  \item \glossario{Modelli di apprendimento}: \glossario{SVM} e \glossario{Regressione Lineare}
  \item \glossario{Java JMX}: uno strumento utilizzato per il monitoraggio di applicazioni sviluppate in \glossario{Java}
  \item \glossario{Java JMeter}: uno strumento utilizzato per verificare il comportamento di \glossario{applicazioni web}
\end{itemize}

\subsection{Aspetti positivi}%
\label{sub:aspetti_positivi}
\begin{itemize}
  \item Le conoscenze di \glossario{Machine Learning} vengono sempre più richieste dalle imprese quindi sarebbe un’opportunità poterle imparare
  \item L’azienda è ampiamente disponibile a fornire la formazione sugli algoritmi e tecnologie di \glossario{Machine Learning}
\end{itemize}

\subsection{Rischi}%
\label{sub:rischi}
\begin{itemize}
  \item Il corso di studi della laurea triennale non copre il \glossario{Machine Learning} quindi si dovrà utilizzare una tecnologia a noi poco familiare
\end{itemize}

\subsection{Conclusione}%
\label{sub:conclusione}
Il progetto ha riscontrato grande interesse nel gruppo e il poter essere formati da una grande \glossario{software house} come Zucchetti ci ha spinto ad optare per questo capitolato.

\end{document}